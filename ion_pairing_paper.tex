\documentclass[journal=jacsat,manuscript=article]{achemso}
\usepackage[utf8]{inputenc}
\usepackage[T1]{fontenc}       % Use modern font encodings
\usepackage{amssymb}
\usepackage[version=3]{mhchem} % Formula subscripts using \ce{}
\usepackage{siunitx}
\usepackage[table]{xcolor}

\usepackage{xr}
\externaldocument[S-]{ion_pairing_SI}

\author{Hassan Srour}
\affiliation[Laboratoire de Chimie de l'ENS de Lyon]{Laboratoire de Chimie UMR CNRS 5182 Ecole Normale Supérieure de Lyon/ Université Claude Bernard Lyon1/ Université de Lyon 46 Allée d'Italie, 69007 Lyon}


\author{Mathieu Leocmach}
\affiliation[Institut Lumière Matière]{Institut Lumière Matière, CNRS UMR 5306, Université Claude Bernard Lyon 1, Université de Lyon, Lyon, 69622 Villeurbanne Cedex, France}
\email{mathieu.leocmach@univ-lyon1.fr}

\author{Martien Duvall Deffo Ayagou}
\author{Thi Thanh-Tam Nguyen}
\affiliation[Laboratoire de Chimie de l'ENS de Lyon]{Laboratoire de Chimie UMR CNRS 5182 Ecole Normale Supérieure de Lyon/ Université Claude Bernard Lyon1/ Université de Lyon 46 Allée d'Italie, 69007 Lyon}

\author{Nicolas Taberlet}
\author{Sébastien Manneville}
\affiliation[Laboratoire de Physique de l'ENS de Lyon]{Laboratoire de Physique, Ecole Normale Supérieure de Lyon/ Université Claude Bernard Lyon1/ Université de Lyon, 46 Allée d'Italie, 69007 Lyon}


\author{Chantal Andraud}
\author{Cyrille Monnereau}
\affiliation[Laboratoire de Chimie de l'ENS de Lyon]{Laboratoire de Chimie UMR CNRS 5182 Ecole Normale Supérieure de Lyon/ Université Claude Bernard Lyon1/ Université de Lyon 46 Allée d'Italie, 69007 Lyon}
\email{cyrille.monnereau@ens-lyon.fr}

%\title{Supramolecular crocodile line controls the rheological properties of polyelectrolytes hydrogels}
\title{Ion pairing controls rheological properties of ``processionary'' polyelectrolyte hydrogels}
%\title{Ion pairing in crocodile line hydrogels controls rheological properties}
%\title{In crocodile line hydrogels ion pairing controls rheological properties}
%\title{Controlling the rheological properties of well-defined polyelectrolytes hydrogels by fine tuning their ion pairing process}


\keywords{ATRP, hydrogel, poly(cations), rheology}

\begin{document}

\begin{tocentry}

%9\,cm by 3.5\,cm
\includegraphics{TOC2.png}

\end{tocentry}

\begin{abstract}
We demonstrated recently that polyelectrolytes with cationic moieties along the chain and a single anionic head are able to form physical hydrogels due to the reversible nature of the head-to-body ionic bond. Here we generate a variety of such polyelectrolytes with various cationic moieties and counterion combinations starting from a common polymeric platform. We show that the rheological properties (shear modulus, critical strain) of the final hydrogels can be modulated over three orders of magnitude depending on the cation/anion pair. Our data fit remarkably well within a theoretical model involving a supramolecular head-to-tail single file between cross-links, akin to the behaviour of pine-processionary caterpillar, the length of the procession being set by the amount of charge dissociation.
\end{abstract}

\section{Introduction}
Since the pioneering work of \citet{Wichterle1960} which established their relevance in a biomedical context, polymer-based hydrogels have never ceased to be a very active field of research\cite{Calo2015,Hoffman2001}. They have recently experienced a burst of interest among the biomedical community as controlled drug delivery cargos or scaffolds for wound healing or tissue repair\cite{Vashist2014,Ratner2004}. Polyelectrolytes are being extensively put to use in this particular context\cite{Rosso2003,BinImran2014}. Their gel formation mechanism often involves reversible-by-nature electrostatic interactions, which can be used for instance to trigger ``smart'' release of bioactive substances\cite{Khare1993,Lockwood2007}. Besides, poly(cationic) gels have been reported to combine scaffold properties for cell’s adherence and growth with antimicrobial activity, and are therefore increasingly used in clinical applications\cite{Hoffman2001,Landers2002}. In this framework, injectability is a key feature, as it provides an easier way to gel delivery in vivo\cite{Tibbitt2016}; thus, reversible shear thinning biocompatible hydrogels are a particularly sought after class of materials\cite{Yu2008}.


Contrary to a small electrolyte for which full ion pairs dissociation is generally achieved in water, polyelectrolytes in solution are in general not fully dissociated, with a substantial fraction of the counterions bound to the polymer\cite{Manning1979}. The resulting net charge of the chain governs the physical properties of the polyelectrolyte solution\cite{Rubinstein1996}, first of all the ability of the polymer to dissolve. Counterion condensation is the physical bounding or adsorption of counterions near the polymer chain. Factors influencing this process have been known since the 1880's but are still not fully understood. \citet{Hofmeister1888} was the first to propose a systematic ranking of ions, based on their propensity to promote the coalescence of egg white. This so-called Hofmeister series has since proved quite universally valid\cite{Zhang2010a}, including multi-charged polymers such as naturally occurring proteins or synthetic polyelectrolytes as well as charged colloidal particles\cite{Schwierz2010,Oncsik2015} or soft matter interfaces in general\cite{Leontidis2014}. However, depending on the nature of the polymer (hydrophilic or hydrophobic, anionic or cationic), the Hofmeister series can be direct (in short, well hydrated ions promote condensation) or reverse\cite{Schwierz2010,Oncsik2015}.

We have recently reported on a highly reversible poly(cation) based hydrogel\cite{Srour2014}, which formation relies on a new concept of supramolecular  electrostatic interaction. Briefly, the atom transfer radical polymerisation (ATRP) is initiated by a phosphonate-terminated ATRP initiator, eventually affording an anion-terminated poly(cationic) polymer. The resulting poly(cationic) polymer, when dispersed in an aqueous medium, stabilizes hydrogel formation through the occurrence of a head(anion)-to-body(cations) supramolecular network. This highly dynamic electrostatic network provides the resulting gel with spectacular mechanical and self-healing properties. Moreover the interaction can be disrupted by various chemical stimuli, such as pH or ionic strength.

In the present report, we investigate the role of the couterion condensation on the mechanical properties of such hydrogels. In a first part we describe how we take advantage of our postfunctionalization approach to play systematically with the nature (aromatic or not) of the cationic repeating unit and the associated halide counterion (\ce{F-}/\ce{Cl-}/\ce{Br-}/\ce{I-}) varied along the Hofmeister series. In the second and third parts, we show respectively the qualitative change in gel formability and the quantitative variations of mechanical properties of the aqueous dispersions obtained from these well-characterized polymers. In a fourth part, we rationalize this behaviour by proposing a microscopic model based on the idea that at low dissociation rates, cross-links (defined as a point where three or more polymers meet) are not separated by a single macromolecule but by several (up to hundreds) polymers in a supramolecular chain. We call this behaviour ``processionary'' in analogy to pine processionary caterpillar (\textit{Thaumetopoea pityocampa}) behaviour\cite{Fabre1916}.%Finally, we discuss our results and possible applications of such smart hydrogels.

\section{Results and discussion}

\subsection{Synthesis and characterizations}

We synthesise the phosphonate terminated polymer hereafter referred to as \ce{PBr} according to our previously reported methodology\cite{Srour2014,Appukuttan2012}. We obtain a well-controlled linear polymer ($Mw= \SI{5614}{\dalton}$, $Mw/Mn = 1.08$; see Supplementary Materials and Supplementary Figure~S\ref{S-fig:sec}), with a degree of polymerisation $N_0=70$. This polymer serves as a common scaffold from which we derive  all studied systems. Nucleophilic addition of N-methylimidazole or N-methylpyrrolidine to a heated solution of \ce{PBr} in THF affords the corresponding poly(imidazolium) and poly(pyrrolidinium) compounds. In these cases, as an inherent consequence of the structure of the starting material, charge neutrality is provided by bromide counter ions, and the polymer will be referred to as \ce{PIm+Br-} and \ce{PPyr+Br-}, respectively (Scheme~\ref{sch:synthesis}).

\begin{scheme}
\includegraphics{synthesis.pdf}
\caption{Synthesis of \ce{PIm+Br-} and \ce{PPyr+Br-} and their intermediates \ce{POH} and \ce{PBr} with $N_0=70$.}
\label{sch:synthesis}
\end{scheme}

We performed anionic metathesis by pouring an aqueous solution of \ce{PIm+Br-} or \ce{PPyr+Br-} into a saturated aqueous solution of the different sodium halides (\ce{NaF}, \ce{NaCl}, \ce{NaI}) (Scheme~\ref{sch:metathesis}). In the cases of \ce{PIm+F-}, \ce{PIm+Cl-} , \ce{PPyr+F-} and \ce{PPyr+Cl-}, we obtain a turbid suspension immediately after addition. After extensive dialysis of the resulting mixture against deionized water and lyophilization we recover in high yields the different \ce{PIm+X-} and \ce{PPyr+X-} (where \ce{X}=\ce{F}, \ce{Cl}, \ce{I}). In order to bring evidence for the efficiency of the ionic metathesis, we submit the resulting materials to negative-mode high-resolution mass spectrometry (HRMS, see Supplementary Materials). We unambiguously assess complete displacement of bromide counterions by the full disappearance of the diisotopic mass peak (89/\SI{91}{\dalton}, Supplementary Figures~S\ref{S-fig:massPImBr} and S\ref{S-fig:massPPyBr}). We further confirm substitution by iodide by the concomitant apparition of a characteristic monoisotopic peak (\SI{121}{\dalton}, Supplementary Figures~S\ref{S-fig:massPImI} and S\ref{S-fig:massPPyI}), while fluoride and chloride anions signals, which are below the detection limits of the spectrometer, could not be observed in the corresponding polymers (not shown).

\begin{scheme}
\includegraphics{metathesis.pdf}
\caption{(a) Anionic metathesis reaction. As above, \ce{Nu+} corresponds either to pyrrolidinium or imidazolium. (b) Summary of the polymers derived from a single batch of \ce{PBr}.}
\label{sch:metathesis}
\end{scheme}

Differential Scanning Calorimetry (DSC) thermograms  of all synthesized polymers also reveals marked differences in their properties, that can only be explained by differences in their couterion features. Thus, in the case polyelectrolytes with fluoride and chloride counterion, we observe no exo or endothermic transition below \SI{200}{\celsius} (featureless thermograms, not shown). By contrast, we observe a broad endothermic peak when iodide and bromide are used as counterions, see Figure~\ref{fig:dsc}. Quite remarkably, we found a similar peak temperature (84-\SI{85}{\celsius}) for \ce{PIm+I-} and \ce{PPyr+I-}. When bromide counterions are present, we observe a significant increase of the peak temperature (114-\SI{115}{\celsius}) but again, with a similar value between \ce{PIm+Br-} and \ce{PPyr+Br-} as shown in Figure~\ref{fig:dsc}. The overall shape and position of the DSC peaks is very reminiscent of previously reported data for various naturally occuring or synthetic poly(electrolytes)\cite{Li2005,Sarmento2006,Ostrowska-Czubenko2009a,Moin2015}. It is generally attributed to desorption of weakly bound water from the polymer network. %Its presence in the polymer samples where bromide and iodide are used as couterions suggests that adsorbed water could play a role in solvating the dissociated ion pairs in the latter. Conversely, its absence for chloride and fluoride counterions suggests that dissociation is not effective because of stronger ion pair interaction.

This difference in DSC profile can be understood by considering the water-binding ability of the different polymers in this study. Since the peak temperature does not depend on the presence of imidazolium or pyrrolidinium, water-binding is probably due to the only part of the size groups able to accept hydrogen bonds from water: the ester function. Following \citet{Zhang2010a}, we propose that these bonds can be enhanced if the involved water molecule is polarized by solvating an anion. The polarization, and thus the water-binding ability will decrease along the direct Hofmeister series $\ce{F-}>\ce{Cl-}>\ce{Br-}>\ce{I-}$. We are able to observe the end of this trend in the shift of peak position between bromide and iodide counterions, whereas in the case of chloride and fluoride, binding is too strong to observe water desorption. This hypothesis is further confirmed below, when looking at the properties of different polymers in aqueous dispersion.

\begin{figure}
\includegraphics{dsc.pdf}
\caption{DSC measurements of polymer samples showing the presence of an endothermic peak. Curves are shifted vertically by arbitrary amounts for clarity. Vertical lines indicate the peak temperatures at \SI{84}{\celsius} for the two iodides in black and \SI{114}{\celsius} for the two bromides in gray.}
\label{fig:dsc}
\end{figure}

\subsection{Water swelling, counterion condensation and solvent quality}
Water-swelling properties turn out to be extremely dependent on the polymer composition. With bromide or iodide conterions, both poly(pyrrolidinium) and poly(immidazolium) polymers afford homogeneous and optically transparent gels upon swelling with deionised water. We observe a limited swelling for \ce{PPyr+Cl-}, resulting in a granular, inhomogeneous gel. With fluoride counterions, or in the case of \ce{PIm+Cl-}, we observe neither swelling nor dissolution. We recover a 8\%wt suspension as a biphasic mixture of its individual components.

The above dependence of solubility on the nature of the counterions follows the direct Hofmeister series. The strong condensation of \ce{F-} on the chain decreases the effective positive charge of the polymer and thus reduces solubility. Conversely, bulkier and softer ions like \ce{I-} are less adsorbed, resulting in more dissociated charges and higher solubility. As a consequence, while dissolution is observed for the four polymers with bromide and iodide counterions, in which ion dissociation takes place to a sufficient extent, the two polymers with fluoride counterions, for which couterion condensation is expected to be strong, cannot be dissolved.

However Hofmeister series alone cannot explain the difference between soluble \ce{PPyr+Cl-} and insoluble \ce{PIm+Cl-}. This constitutes a strong indication that counterions are more strongly condensed near imidazolium than near pyrrolidinium. Indeed, although both cationic in nature, pyrrolidinium and imidazolium ions have markedly different properties. Because of their aromatic ring, imidazolium ions are particularly prone to promote a variety of supramolecular interactions which strongly contribute to their physicochemical properties, such as \ce{\pi+-\pi} or anion$-\pi^+$.  More recently, \ce{\pi+-\pi+} interactions have also been identified as a distinctive driving force for imidazolium dimerization\cite{Geronimo2011}. Because of strong intercoulombic repulsion, \ce{\pi+-\pi+} interactions are primarily weaker than their \ce{\pi-\pi} counterparts. However it has been established that, in the presence of couterions, the stabilizing effect of \ce{\pi+-\pi+} interaction could reach magnitudes largely exceeding that of the latter\cite{Geronimo2011}. In particular, dimerization of imidazolium-chloride ion pairs have been the object of recent studies, and it has been shown that the involvement of the negatively charged counterion was essential in maximizing the stabilization of the \ce{\pi+-\pi+} complex by minimizing coulombic repulsion between the imidazolium moieties\cite{Matthews2014,Gao2015a}. It is likewise well documented that covalent incorporation of interacting groups within a polymer chain strongly favours intramolecular interactions between these groups, by increasing their so-called effective molarity\cite{Li2003,Mulder2004,Huerta2013}. It is therefore not surprising that both effects participate in making the imidazolium-anion interaction stronger than pyrrolidinium at the detriment of solubility.

Such an insolubility in absence of charge dissociation indicates that the chain is in poor solvent. Locally, the monomers are condensed in collapsed globules to minimize the contacts with water molecules, and solubility on larger scales can only be achieved if the polymer bears enough charges\cite{Khokhlov1980,Raphael1990}. To estimate the $\Theta$-temperature, that delimit bad and good solvent behaviour, we held the two fluoride polymers in water at boiling temperature overnight. No dissolution or swelling was observed, implying that $\Theta>\SI{100}{\celsius}$. %Nonetheless, due to charge dissociation, five of our polyelectrolytes are soluble at \SI{25}{\celsius}.


%\begin{figure}
%\caption{pictures of 8\%wt water solutions of \ce{PIm+X-} and \ce{PPyr+X-} in water.}
%\label{fig:pictures}
%\end{figure}

\subsection{Rheological measurements}
\begin{figure}[t]
\includegraphics{strain.pdf}%
\caption{Storage modulus $G^\prime$ ($\bullet$) and loss modulus $G^{\prime\prime}$ ($\square$) measured through oscillatory shear experiments plotted against the strain amplitude $\gamma$. The moieties change with rows and the counterions with columns. The fixed frequency is $f=\SI{1}{\hertz}$. All samples are at 8\%wt. The gray area limits of the linear, plastic and shear thinning regimes.}%
\label{fig:strain}%
\end{figure}

We perform rheological studies of the water-swelling materials. Briefly, data are recorded at \SI{25}{\degree} with an AR 1000 rheometer (TA Instruments) in a cone-plate geometry of radius \SI{40}{\milli\metre}, an angle of \SI{2}{\degree} and a truncation of \SI{58}{\micro\metre}\cite{Macosko1994,Larson1999}. We place the sample on the plate, and then lower the cone to the measuring position, spreading the sample in the process. We remove excess material, so that the sample exactly fills the gap. Due to Weissenberg effect at high shear rate, we apply no preshear before measurement. To minimize water absorption, we cover the geometry with a solvent trap, using light mineral oil as a liquid seal between the rotor and the cap. After one minute of equilibration we perform an oscillatory frequency sweep at small amplitude (strain amplitude $\gamma=0.1\%$, see Supplementary Figure~S\ref{S-fig:frequency}) and then an oscillatory strain sweep at fixed frequency ($f=\SI{1}{\hertz}$). For fluoride counterions as well as \ce{PIm+Cl-}, the rheological profile is that of pure water which confirms the visual observation. However as shown in Figure~\ref{fig:strain}, all other samples are solidlike ($G^\prime \gg G^{\prime\prime}$) at small strain and flow at large strains ($G^{\prime\prime} \gg G^\prime$) with a steep decrease of the moduli. We confirm solidlike behavior at low strain for all accessible frequencies (see supplementary material S6 Fig. 1). We checked that the linear mechanical properties are unchanged by the flow history given a few minutes of rest.

The mechanical properties at low strain evolve with the factors influencing counter-ion condensation. Imidazolium based gels are much weaker than their pyrrolidinium counterpart. Comparison between \ce{PIm+Br-} and \ce{PPyr+Br-} is particularly illustrative of this trend, as their $G^\prime$ value (13 and \SI{8100}{\pascal}, resp.) differ by almost three orders of magnitude. On the bottom row of Figure~\ref{fig:strain} we can compare the rheological responses of Poly(pyrrolidinium)-based hydrogels for the three counterions allowing dissolution in water. Gels with bromide and iodide counterions show roughly the same modulus value at vanishing strain for both $G^\prime$ and $G^{\prime\prime}$. By contrast, the gel with chloride counterions is weaker by more than one order of magnitude. In other words, counterion condensation is correlated to softer gel. This trend is confirmed between \ce{PIm+Br-} and \ce{PIm+I-}, the former being an order of magnitude softer than the later.

Moderately charged \ce{PIm+I-} and \ce{PPyr+Cl-} show a plateau in both moduli at low strain corresponding to the linear regime of the material; an overshoot of $G^{\prime\prime}$ and a downward slope of $G^\prime$ at intermediate strains corresponding to the onset of plasticity~\cite{Hyun2011}; and a decrease of both moduli at large strains, steeper for $G^\prime$ than for $G^{\prime\prime}$, indicating shear thinning. The linear regime is either extremely narrow or non existent in heavily charged \ce{PPyr+Br-} and \ce{PPyr+I-}. By contrast the lightly charged \ce{PIm+Br-} displays a broad intermediate regime (tree decades of strain) between the linear regime and the crossing of the moduli. In any case, the end of the linear regime corresponds to a strain much smaller than 1.

In the following we show that the correlation between softness and counterion condensation can be explained on the basis of a microscopic model of interchain interactions.

%We define the critical strain value $\gamma_c$ between these two regimes when $G^{\prime\prime}$ reaches a maximum. We checked that the linear mechanical properties are unchanged by the flow history given a few minutes of rest. %To understand the influence of the ion pairing on the mechanical properties, we compute the cohesive energy (equation 1) that should probe the strength of the physical supramolecular bonds.15,16



%In Figure~\ref{fig:strain}a we compare the rheological responses of Poly(pyrrolidinium)-based hydrogels for the three counterions allowing dissolution in water. Gels with bromide and iodide counterions show roughly the same mechanical behavior in the linear regime but \ce{PPyr+I-} yields at slightly higher strains ($\gamma_c=4\%$ instead of $3\%$). By contrast, the gel with chloride counterions is weaker by more than one order of magnitude. In other words, counterion condensation is correlated to softer gel.

%We also compare the rheology of \ce{PIm+X-} and \ce{PPyr+X-} with the same counterion \ce{X} in Figure~\ref{fig:strain}b,c. For both iodide (Figure~\ref{fig:strain}b, X=I) and bromide (Figure~\ref{fig:strain}c, X=Br), imidazolium based gels are much weaker than their pyrrolidinium counterpart with a broader linear regime. Comparison between \ce{PIm+Br-} and \ce{PPyr+Br-} is particularly illustrative of this trend, as their $G^\prime$ value (13 and \SI{8100}{\pascal}, resp.) and $\gamma_c$ value (804\% and 1.6\%, resp.) differ by almost three orders of magnitude. Here also, gel toughness is related to charge dissociation efficiency. In the following we show that this correlation can be explained on the basis of a microscopic model of interchain interactions.

\subsection{Processionary model}

\begin{figure}
\includegraphics{dissociation.pdf}
\caption{Processionary model. (a) Sketch of the network in the case of a procession of size $n=3$ between cross-links (gray disks) and a persistence length $r_\mathrm{src}\approx 3D$. Empty circles are not individual monomers but electrostatic blobs of size $D$. Electrostatic blobs containing an anionic head are shown as black filled circles. (b) Relation between the number of monomers between uncondensed charges and the number of chains between cross-link points. The solid line corresponds to $n = A^2$. Full circles are obtained from (Eq.~\ref{eq:A}). In the case of \ce{PIm+Br-}, we are not in the validity regime of (Eq.~\ref{eq:A}), and we have to extrapolate the solid line to estimate $A$ (dotted arrow and open symbol).}
\label{fig:dissociation}
\end{figure}

We have previously demonstrated that, due to the opposite charges of the phosphonate head, head-to-body ionic bonds are possible~\cite{Srour2014}. On the one hand, if at least two foreign heads attach to the same body, we obtain an effective cross-link point. When the probability of such a configuration is non-zero, we obtain a physically cross-linked gel. This is the ideal situation that we described in our previous paper.

On the other hand, if every chain has a single foreign head attached to its body, every polymer is linked to two others in a single file\bibnote{We neglect the possibility of the two charges of the phosphonate head to attach to two separate chains}. In addition, if we suppose that in order to minimize the inter-chain repulsion between charged cationic groups, heads are preferentially attached to the tail of their neighbour, we obtain a linear chain of effective polymerisation index $n N_0$ where $n$ is the number of polymers in the supramolecular queue. This behaviour is somewhat evocative of that of the pine processionary caterpillar memorably described by \citet{Fabre1916}: 
``They proceed in single file, in a continuous row, each touching with its head the rear of the one in front of it. [...] No Greek theoria winding its way to the Eleusinian festivals was ever more orderly. Hence the name of Processionary given to the gnawer of the pine.''



We thus have two limiting cases: (i) every chain has at least two heads attached and we have roughly $N_0$ monomers between cross-link points; (ii) every chain has at the most a single head attached and we have isolated supramolecular chains in the system. The former case should be observed when a significant number of ion-pairs along the polymer chain are dissociated providing a significant probability for multiple phosphonate heads binding on a single polymer chain. Conversely, the latter case should be observed at low charge dissociation. In between these two limiting cases, we should observe cross-link points separated by processions of $n N_0$ monomers as sketched in Figure~\ref{fig:dissociation}a.

As we will discuss in the following, all rheological features of the hydrogels studied here can  be rationalised on the basis of this ``processionary'' model. We base our analysis on two main observables, see Table~\ref{tab:results}: (i) the value of the shear modulus at small amplitude $G^\prime(\gamma\rightarrow 0)$, i.e. the elasticity of the undamaged gel network, and (ii) the critical strain $\gamma_c$ corresponding to the maximum of $G"$, that we consider as representative of the end of the linear regime and of the first broken cross-links\cite{Hyun2011}. By using these two parameters, we show that it is possible, with reasonable assumptions, to estimate microscopic parameters as the average number of chains between cross-links $n$, the head-to-body bonding energy $E_c$ or the number of monomers between dissociated ion pairs $A$.


\subsubsection{Conformation at rest}

Let us first recall the structure of a polyelectrolyte chain as described by the scaling theory of \citet{Rubinstein1996}. In absence of strain, a linear polyelectrolyte chain is organised at various scales.

Below the thermal correlation length $\xi_T$ monomer interactions dominate over the thermal energy, thus chains behave ideally and we need to take into account the geometry of the polymer. Due to the relatively large lateral moieties ($s=11$ non-hydrogen atoms) with respect to the distance between them (2 carbons), the polymer can be considered as a comb polymer or a bottlebrush. According to \citet{Feuz2005} the persistence length is thus longer than the distance $b$ between two monomers and corresponds to the thickness $d=s^{2/3} a^{4/3}b^{-1/3}$ of the brush in ideal conditions, where $a\approx\SI{150}{\pico\metre}$ is the distance between two non-hydrogen atoms. Taking $b=a\sqrt{3}$, we obtain $d = \left(s/3\right)^{2/3}b \approx 2.4 b$. We note $g_c=d/b$ the number of monomers in the persistence length.

The density in thermal blobs is set by the reduced temperature $\tau= 1-T/\Theta$ such that $(g_T/g_c)/\xi_T^3 = \tau/d^3$ and their structure is a persistent random walk $\xi_T = d (g_T/g_c)^{1/2}$. These conditions yields $\xi_T = d/\tau$ and $g_T = g_c/\tau$.

At a scale $D$, the electrostatic energy balances the surface energy of the collapsed polymer. We name $g_e$ the number of monomers in the electrostatic blob. If there is a charge every $A$ monomers and there are $g_e/A$ charges in the electrostatic blob. The surface tension is proportional to the number of thermal blobs exposed on the surface $k_\mathrm{B}T/\xi_T^2$, with $k_\mathrm{B}$ is the Boltzmann constant and $T$ the temperature. Thus the balance of energies writes:
\begin{equation}
\left(\frac{D}{\xi_T}\right)^2 = \left(\frac{g_e}{A}\right)^2 \frac{\ell_\mathrm{B}}{D},
\label{eq:electrosurface}
\end{equation}
where $\ell_\mathrm{B} = e^2/(4\pi\epsilon k_\mathrm{B}T)$ is the Bjerrum length with $e$ the elementary charge, $\epsilon$ the dielectric constant of the solvent. In water $\ell_\mathrm{B} \approx \SI{0.7}{\nano\metre}$. Since thermal blobs fill the volume of the electrostatic blob $D = \xi_T \left(g_e/g_T\right)^{1/3}$, we obtain
\begin{equation}
g_e = \frac{A^2}{u} \tau, \qquad D = b\left(\frac{A^2}{u}g_c^2\right)^{1/3},\qquad \text{with }u = \ell_\mathrm{B}/b.
\end{equation}

We define the extension parameter $B$ as the ratio between the length of a chain of $g_e$ monomers fully extended and the actual size of the electrostatic blob: 
\begin{equation}
B = \frac{g_e b}{D} = \left(\frac{A^2}{u g_c}\right)^{2/3} \tau
\end{equation}

At constant solvent quality and polymer architecture, $B$ is thus monotonically related to the amount of counterion condensation. In the following we will estimate $B$ from rheological measurements.

Electrostatic repulsion decays with a characteristic length $r_\mathrm{src}$, the screening length. Following\cite{Rubinstein1996} this screening length is set by the concentration of dissociated counterions and the concentration of $c_s$ added salts:
\begin{equation}
r_\mathrm{src} = \left(\frac{B}{cb}\right)^{1/2} \left(1 + Bu \frac{c_s}{c}\right)^{-1/2}.
\end{equation}
Between $D$ and $r_\mathrm{src}$ the electrostatic blobs are organised in a linear rod containing $g_\mathrm{src}$ monomers. The length of this rod is $B$ times shorter than if the electrostatic blobs were extended: 
\begin{equation}
g_\mathrm{src} = B \frac{r_\mathrm{src}}{b} = \left(\frac{B^3}{cb^3}\right)^{1/2} \left(1 + Bu \frac{c_s}{c}\right)^{-1/2}.
\label{eq:gscr}
\end{equation}

The last length scale is the correlation length $\xi$. Between $r_\mathrm{src}$ and $\xi$ the chain forms a self avoiding walk of persistence length $r_\mathrm{src}$. Above the correlation length the polyelectrolyte chain forms a random walk of correlation blobs containing $g=c\xi^3$ monomers.

\subsubsection{From modulus to procession length}

If between two crosslinks or entanglements the ``procession'' of polymer chains contains more monomers than $g$ or $g_\mathrm{src}$ it performs a random or self avoiding walk respectively. Each chain is thus an entropic spring of constant stiffness $k_\mathrm{B}T$. The number density of chains is $c/N$ with $c$ the monomer concentration and $N$ the number of monomers between cross-links or entanglements. Therefore, the shear modulus at low strain is given by:
\begin{equation}
G = \frac{c}{N}k_\mathrm{B}T,
\label{eq:G}
\end{equation}
where $c$ is the monomer number density. Because of the different weights of the pyrrolidinium and imidazolium moieties and of the different counterions, $c$ is not constant as we chose to conduct our experiments at constant weight fraction $w$ of polymer (generally used to quantify gelation ability of a given gelator). This explains why the respective modulus or critical strains of \ce{PPyr+I-} (heavier, lower number density) and \ce{PPyr+Br-} are in reverse order with respect to the Hofmeister series. Taking into account the molecular mass $M$ of each polymer and  knowing the density $d$ of the solvent, we obtain
\begin{equation}
n = \frac{\mathcal{N_A}}{M} w d \frac{k_\mathrm{B}T}{G}.
\label{eq:n}
\end{equation}

We find that the number $n$ of chains between cross-link point goes from 1 in \ce{PPyr+I-} and \ce{PPy+Br-} to 880 in \ce{PIm+Br-}, following the \textit{a priori} ranking of charge dissociation, see Table~\ref{tab:results}. This larger cross-link ratio is consistent with a higher probability of attaching two or more heads on a highly dissociated body.

\begin{table}
\begin{tabular}{l|SS|SSSSS}
& {$G^\prime(\gamma\rightarrow 0)$} & {$\gamma_0$} & {$n$} & {$B$} & \multicolumn{2}{c}{$A$}\\
&	{\si{\pascal}} & {\%} & & &{$\tau=0.2$}& {$\tau=1$}\\\hline&&&&&\\[-10pt]
\ce{PIm+Br-}	& 13	&	1.1	&	860	&	740	&	1370	&	410\\
\ce{PIm+I-}		& 78	&	1.8	&	122	&	104.8	&	317	&	94\\
\ce{PPyr+Cl-}	& 270 	&	2.8	&	50.6	&	55.7	&	197	&	59\\
\ce{PPyr+Br-}	& 9800 	& 	0.3	&	1.15	&	2.91	&	21	&	6.5\\
\ce{PPyr+I-}	& 8800	&	0.0	&	1.08	&	2.63	&	19	&	6.0\\
\end{tabular}
\caption{Summary of rheological measurements and microscopic values deduced from the model. Gray background indicates inconsistent values obtained either by (Eq.~\ref{eq:Ec}) or by (Eq.~\ref{eq:A}) outside of their respective validity domain.}
\label{tab:results}
\end{table}

\subsubsection{Limit of the linear regime}

As the material is strained, the random walk is stretched, then the self-avoiding walk. When the correlation blobs are fully stretched, the chain is a linear assembly of electrostatic blobs. Stretching the chain further means exposing more thermal blobs to the solvent, a process that cannot be reduced to spring of constant stiffness. This condition is thus the limit between the linear and non linear regimes.

For all our samples, the linear regime is narrow $\gamma_0\approx 10^{-2}\ll 1$ if existing. It means that procession is just long enough to start performing the self-avoiding walk. Incidentally, this implies that the procession is too short to perform the random walk. In other words the number $N$ of monomer between crosslinks is comparable to $g_\mathrm{scr}$. Using equation (\ref{eq:gscr}), this condition translates into
\begin{equation}
\left(\frac{B}{B_0}\right)^3 = 1 + \frac{B}{B_s}
\label{eq:solveB}
\end{equation}
with $B_s = c/u c_s$ and $B_0 = b c^{1/3} N^{2/3} = bc \left(k_\mathrm{B}T/G\right)^{2/3}$. Here, the ``salt'' are the phosphonate heads, that is to say two charges per chain. Thus $c_s/c = 2/N_0$ and $B_s = N_0/2u \approx 23$.

Analysing Equation (\ref{eq:solveB}), we identify two physically relevant limit cases: (i) if $B \ll B_s$ (low salt condition) then $B = B_0$; (ii) if $B \gg B_s$ (high salt condition) then $B = \left(B_0^3/B_s\right)^{1/2} > B_0$.

Numerical solutions of Equation (\ref{eq:solveB}) are given in Table~\ref{tab:results}. For the same quality of solvent $B$ is monotonically related to the amount of counterion condensation. We thus confirm that our samples are sorted by decreasing counterion condensation (decreasing $B$). More quantitatively, the number $A$ of monomers between dissociated charges is
\begin{equation}
A = g_c u^{1/2} \left(\frac{B}{\tau}\right)^{3/4}
\end{equation}
We know that $\Theta>\SI{100}{\celsius}$ so $0.2<\tau<1$.

\subsubsection{Bounding energy}

Considering that between two cross-links a procession is an entropic spring in its linear domain, the mechanical energy stored under a strain $\gamma$ is $E = \gamma^2 k_\mathrm{B}T$. If we suppose that ionic bonds break before the complete extension of the procession, we evaluate the head-to-body bonding energy to
\begin{equation}
E_c = \gamma_c^2 k_\mathrm{B}T.
\label{eq:Ec}
\end{equation}

For \ce{PIm+Br-} we measure $\gamma_c = 800\%$ and thus $E_c \approx 65 k_\mathrm{B}T$ per head-to-body bond or \SI{160}{\kilo\joule\per\mol}. This value is much higher than the $E_i\approx\SI{5}{\kilo\joule\per\mol}$ typical of ionic bonds in water\cite{Schneider1992}, even when adding up the two bonds of the bivalent phosphonate head\bibnote{Since the two minus charges are borne by the same atom, chelation effect may enhance bounding energy, but not by an order of magnitude}. However, one has to keep in mind that the chain is in bad solvent, thus forming locally compact blobs of low water content. Therefore, the medium surrounding the ionic bond is not water but the hydrophobic chains. Such a low polarity microenvironement is well known to enhance otherwise weak electrostatic interactions in protein folding or engineered self-assembly\cite{Rehm2010}. Quantitatively, the bonding energy of an ionic bond is inversely proportional to the relative dielectric constant $\epsilon_r$, therefore we need to suppose a local $\epsilon_r$ around 5 (between 2-3 typical of water-insoluble polymers and 80 for water) to recover the large bonding energy that we measure.

However for all other samples the critical strain is between 2\% and 20\%, leading to inconsistent values for the bonding energies ranging between $10^{-3}k_\mathrm{B}T$ and $10^{-1}k_\mathrm{B}T$ per bond, see Table~\ref{tab:results}. Therefore, the end of the linear regime cannot correspond to bond breaking but rather to the point where extended chains become nonlinear.


\subsubsection{From critical strain to charge dissociation}

The spatial conformation of a polyelectrolyte chain in semi-dilute, low salt solution is a persistent random walk where the persistence length is the screening length $r_\mathrm{src}$ due to the dissociated counter ions\cite{Rubinstein1996}. In bad solvent, the persistent random walk is made of electrostatic blobs of diameter $D$ containing $g_e$ monomers, see Figure~\ref{fig:dissociation}a:
\begin{align}
g_e &= \frac{A^2}{u}\tau\label{eq:ge},\\
D &= b \left(\frac{A^2}{u}\right)^{1/3}\label{eq:D}.
\end{align}
Here $A$ is the number of monomers between dissociated ion pairs, $\tau = 1 - T/\Theta$ the reduced temperature and $u = \ell_\mathrm{B}/b \approx 2.7$ is the ratio of the Bjerrum length in water $\ell_\mathrm{B}\approx \SI{0.7}{\nano\metre}$ and monomer size $b\approx \SI{0.26}{\nano\metre}$. 

Extending these electrostatic blobs means exposing more monomers to the bad solvent and typically requires energies at least an order of magnitude higher than for the extension of the persistent random walk. We will thus consider that the exit from the linear domain of the material corresponds to the full extension of the persistent random walk:
\begin{equation}
\gamma_c = \frac{R_0}{R} - 1
\end{equation}
where $R$ and $R_0$ are respectively the end-to-end distance and the contour length of the persistent random walk.

A segment of the persistent random walk is $r_\mathrm{scr}$ long and thus contains $g_\mathrm{scr} = g_e r_\mathrm{scr}/D$ monomers. We thus have
\begin{equation}
\left. \begin{array}{ll}
R &= r_\mathrm{scr} \left(\frac{N}{g_\mathrm{scr}}\right)^{1/2}\\
R_0 &= r_\mathrm{scr} \frac{N}{g_\mathrm{scr}}
\end{array}\right\rbrace\quad
\gamma_c = \left(\frac{N}{g_\mathrm{scr}}\right)^{1/2} -1\label{eq:gamma0}.
\end{equation}

\citet{Rubinstein1996} express the screening length as
$r_\mathrm{scr} = \left(A^2 / u\right)^{1/3} \left(\tau /cb \right)^{1/2}$. Combining with (Eq.~\ref{eq:ge}) and (Eq.~\ref{eq:D}) we deduce
\begin{equation}
g_\mathrm{scr} = \frac{A^2}{u} \left(\frac{\tau^3}{c b^3}\right)^{1/2}\label{eq:gscr}.
\end{equation}

From (Eq.~\ref{eq:gamma0}) and (Eq.~\ref{eq:gscr}) we can finally relate the measured critical strain to the microscopic parameter $A$:
\begin{equation}
A = \frac{\left(N u\right)^{1/2}}{\gamma_c+1}\left(\frac{c b^3}{\tau^3}\right)^{1/4}.
\label{eq:A}
\end{equation}

The only unknown here is the reduced temperature $\tau$, possibly different between poly\-(pyrrolidinium) based and poly\-(immidazolium) based polymers. We know that the undissociated chain has a $\Theta$ temperature higher than \SI{100}{\celsius}, thus $0.2<\tau<1$. In addition, we must have $A\geq 1$, which implies that $\tau\leq 0.235$ through (Eq.~\ref{eq:A}), corresponding to $\Theta\approx\SI{116}{\celsius}$. This limit value yields $A=1$ for both \ce{PPyr+I-} and \ce{PPyr+Br-}, indicating full dissociation of the cationic moieties. By contrast \ce{PPyr+Cl-} has a charge every 7 monomers. Supposing the same $\tau$, \ce{PIm+I-} has a dissociated ion pair every 10 monomers, see Table~\ref{tab:results}. Of course analysing \ce{PIm+Br-} based on the value of $\gamma_c$ gives an inconsistent value $A=3.4$ indicating that this hydrogel does not belong to the same regime.

\subsubsection{Charge dissociation and cross-linking ratio}

Finally we cross-check our model by linking the ion pair dissociation and the amount of physical cross-links by an independent argument. A second charge dissociated on a chain creates a possible binding site for a second head-to-body association and thus a cross-link. Therefore the probability to have a cross link is proportional to the probability of having a second charge dissociation. If we assume that the probability for an ion pair to dissociate is independent of the number of dissociated pairs already present on the polymer, we can describe the probability of having $k$ dissociated pairs on a polymer by a Poisson distribution of mean $\lambda = N_0/A$. Therefore, the cross-link probability $1/n$ is proportional to $\lambda^2$, yielding $n \sim A^2$.

As shown in Figure~\ref{fig:dissociation}b and the last two columns of Table~\ref{tab:results} this relation describes very well our data with a prefactor close to 1. We can extrapolate this law for the case of \ce{PIm+Br-} and obtain $A = 28$. Inverting (Eq.~\ref{eq:A}) yields $R_0/R - 1 = 860\%$, indeed higher than the observed $\gamma_c\approx 804\%$. This confirms that in the \ce{PIm+Br-} strain sweep $\gamma_c$ actually results from the strength of the ionic bonds.

%We can now estimate the hydrophobic interaction energy $E_h$ by considering that to separate two chains two interfaces with water of area $\pi D^2/4$ have to be created. Surface tension is $\tau^2 k_\mathrm{B}T / b^2$\cite{Rubinstein1996}. Combining with (Eq.~\ref{eq:D}) we find
%\begin{equation}
%\frac{E_i}{k_\mathrm{B}T} = \frac{\pi}{2}\tau^2\left(\frac{A^2}{u}\right)^{2/3} \approx 4
%\end{equation}






%\subsection{Discussion}
%
%The above dependence of the gel parameters on the nature of the counterions could be rationalized on the basis of ion-pairing processes. The unsolubility/unswellability of \ce{PIm+F-}, \ce{PPyr+F-} and \ce{PIm+Cl-} suggest that ion pairs are not efficiently dissociated in aqueous medium, thus hindering the solvation process. By comparison, \ce{PIm+Br-}, \ce{PIm+I-}, \ce{PPyr+Br-} and \ce{PPyr+I-} form homogeneous materials when suspended in water, which one can correlate with efficient dissociation of the halide counterions from the polymer main chain is fully consistency with decreased ionic bond strength. Such dissociation, in turns makes it possible for the ``hard'' phosphonate group on each polymer chain's end to interact with the ``free'' cationic groups of the side chains, which we showed in a previous paper to be at the origin of the gel formation mechanism\cite{Srour2014}.
%
%The similarity in gel parameters observed for \ce{PPyr+Br-} and \ce{PPyr+I-} ($G^\prime\approx\SI{8500}{\pascal}$) suggests that full ion pair dissociation is achieved in the latter, or at least that this dissociation is enough in both cases to provide maximal interaction between the terminal phosphonate and the pyrrolidinium side groups. Conversely the significantly different behavior between \ce{PIm+I-} ($G^\prime= \SI{78}{\pascal}$) and \ce{PIm+Br-} ($G^\prime= \SI{13}{\pascal}$) indicates that dissociation is only partial and occurs to a lesser extent in the latter than in the former. The same conclusion can be drawn for \ce{PPyr+Cl-}, in which the elastic modulus ($G^\prime= \SI{270}{\pascal}$) value drops by more than one order of magnitude as compared to its bromide and iodide counterparts.
%
%As already mentioned (vide supra) another striking feature of these gels is that in similar conditions poly(pyrrolidinium) based gels are systematically much stronger than those formed with poly(imidazolium). Indeed, although both cationic in nature, pyrrolidinium and imidazolium ions have markedly different properties: whereas pyrrolidinium features a ``hard'' positive charge very localized on the quaternary nitrogen, imidazolium has a much more delocalized ``softer'' charge; consequently, according to the Hard and Soft Acid and Bases (HSAB) theory, stronger association is expected for the latter with the ``soft'' halide counterions\cite{Goossens2009}. As a consequence, for a given halide, dissociation is less favored in the case of poly(imidazolium), as clearly illustrated by the difference in solvation between \ce{PIm+Cl-} and \ce{PPyr+Cl-}. Conversely, association of the lateral cations with the ``hard'' phosphonate moieties at the polymer chain’s end is expected to be stronger with pyrrolidinium, hence a systematically larger elastic modulus. The interplay of these two effects contributes in making poly(pyrrolidinium) based gels much stronger than their poly(imidazolium) counterparts; comparison between \ce{PIm+Br-} and \ce{PPyr+Br-} is particularly illustrative of this trend, as their $G^\prime$ value (13 and \SI{8100}{\pascal}, resp.) differ by almost three orders of magnitude.

\section{Conclusion}
With this work we demonstrated the versatility of our ``head-to-body'' electrostatic approach in the fabrication of hydrogels with readily tunable rheological properties. Thus, depending on the nature of the nucleophile used in the quaternarization step of the polymer, but also on that of the counterion which can be modified in the course of its purification process, we showed that it is possible to manipulate almost at will the gel formability, the density of physical cross-links and the respective magnitudes of the shear modulus and of the shear stress. 

Strongly interacting anions (i.e. small and hard halides, like fluoride) and aromatic cations favour counterion condensation, resulting in too few charges to allow dissolution and gel formation. The hardest gels with the narrowest linear domain are obtained with aliphatic cationic moiety and counterion at the other end of the Hofmeister series that are less bound to the polymer. This gives rise to maximal charge dissociation in water, facilitates charge pairing of the peripheral, iterative cations with the terminal anion and thus affords a very high density of cross-links. In between, in the case of aliphatic cations with a counterion in the middle of the Hofmeister series (\ce{Cl-}) or in the case of aromatic cations and poorly interacting anion, charge dissociation is limited, polymers associate in processions and cross-links are few. We thus obtain soft, highly deformable gels.

To conclude, our procedure yields robust, highly tunable hydrogels from short, linear polymer chains and in the absence of any additive, which could find interesting applications especially in the context of biomaterials. Most importantly, systematic comparisons between the different poly(electrolytes) investigated in this study were used to establish a model linking the microsctructure of the gel to ion pair dissociation efficiency on the individual polymer chains. We hope that this model will join the procession toward future works in supramolecular assemblies.

\begin{acknowledgement}
The authors are grateful for the financial support of Université de Lyon through the program ``Investissements d'Avenir'' (ANR-1 1-IDEX-0007). S.M. and N.T. acknowledge funding from the European Research Council under the European Union's Seventh Framework Program (FP7/2007–2013)/ERC grant agreement No.258803. The Graphical TOC entry is derived from a picture by Arturo Reina, CC BY-SA 3.0, \url{https://commons.wikimedia.org/w/index.php?curid=282898}.
\end{acknowledgement}

\begin{suppinfo}

Materials and methods, detailed synthesis and characterizations of gels, GPC, HRMS and DSC data, calculated cohesion energy, and frequency sweep curves.

\end{suppinfo}

\bibliography{PALSE.bib}

\end{document}