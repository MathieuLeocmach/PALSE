\documentclass[journal=jacsat,manuscript=article]{achemso}
\usepackage[utf8]{inputenc}
\usepackage[T1]{fontenc}       % Use modern font encodings
\usepackage{amssymb}
\usepackage[version=3]{mhchem} % Formula subscripts using \ce{}
\usepackage{siunitx}



\author{Hassan Srour}
\author{Martien Duvall Deffo Ayagou}
\affiliation[Laboratoire de Chimie de l'ENS de Lyon]{Laboratoire de Chimie UMR CNRS 5182 Ecole Normale Supérieure de Lyon/ Université Claude Bernard Lyon1/ Université de Lyon 46 Allée d'Italie, 69007 Lyon}

\author{Mathieu Leocmach}
\affiliation[Institut Lumière Matière]{Institut Lumière Matière, CNRS UMR 5306, Université Claude Bernard Lyon 1, Université de Lyon, Lyon, 69622 Villeurbanne Cedex, France}
\email{mathieu.leocmach@univ-lyon1.fr}

\author{Thi Thanh-Tam Nguyen}
\affiliation[Laboratoire de Chimie de l'ENS de Lyon]{Laboratoire de Chimie UMR CNRS 5182 Ecole Normale Supérieure de Lyon/ Université Claude Bernard Lyon1/ Université de Lyon 46 Allée d'Italie, 69007 Lyon}

\author{Nicolas Taberlet}
\author{Sebastien Manneville}
\affiliation[Laboratoire de Physique de l'ENS de Lyon]{Laboratoire de Physique, Ecole Normale Supérieure de Lyon/ Université Claude Bernard Lyon1/ Université de Lyon, 46 Allée d'Italie, 69007 Lyon}


\author{Chantal Andraud}
\author{Cyrille Monnereau}
\affiliation[Laboratoire de Chimie de l'ENS de Lyon]{Laboratoire de Chimie UMR CNRS 5182 Ecole Normale Supérieure de Lyon/ Université Claude Bernard Lyon1/ Université de Lyon 46 Allée d'Italie, 69007 Lyon}
\email{cyrille.monnereau@ens-lyon.fr}

\title{Controlling the rheological properties of well-defined polyelectrolytes hydrogels by fine tuning their ion pairing process}

\keywords{ATRP, hydrogel, poly(cations), rheology}

\begin{document}

\begin{tocentry}

9\,cm by 3.5\,cm

\end{tocentry}

\begin{abstract}
Starting from a common polymeric platform with lateral bromine substitution and phosphonate end group, it is possible, through two subsequent synthetic steps to generate a variety of poly(ionic liquids) with various cation/anion pair combinations. Upon water addition, the behavior of the resulting hydrogels is seen to be strongly dependent on the polymer structure, illustrating the possibility to control the gel formation properties by fine tuning of the intra and intermolecular ion pairing process.
\end{abstract}

\section{Introduction}
Since the pioneering work of Wichterle et al.,1 which established their relevance in a biomedical context, polymer based hydrogels have never ceased to be a very active field of research.2, 3 They have recently experienced a burst of interest among the biomedical community as controlled drug delivery cargos or scaffolds for wound healing or tissues repair.4 Polyelectrolytes are being extensively used in this particular context.5 Their gel formation mechanism often involves reversible-by-nature electrostatic interactions, which can be used for instance to trigger ``smart'' release of bioactive substances.6 Besides, poly(cationic) gel have been reported to combine scaffold properties for cell’s adherence and growth with antimicrobial activity, and are therefore increasingly used in clinical applications.3, 7 In this framework, injectability is a key feature, as it provides an easier way to gel delivery in vivo; thus, reversible shear thinning biocompatible hydrogels are a particularly sought after class of materials.8
We have recently reported on a highly reversible poly(cation) based hydrogel,9 which formation relies on an new concept of supramolecular  electrostatic interaction. Briefly, the atom transfer radical polymerisation (ATRP) is initiated by a phosphonate terminated ATRP initiator, eventually affording an anion terminated poly(cationic) polymer. The resulting poly(cationic) polymer, when dispersed in an aqueous medium, stabilizes hydrogel formation through the occurrence of a head(anion)-to-body(cations) supramolecular network. This highly dynamic electrostatic network provides the resulting gel with spectacular mechanical and self-healing properties. Moreover the interaction can be disrupted by various chemical stimuli, such as pH or ionic strength.
In the present report, we show that the mechanical/rheological properties of such smart hydrogels can be tuned to an extremely large extent by playing with the nature (hardness/softness) of the cationic repeating unit and the ion-paring strength of the associated counterion (\ce{F-}/\ce{Cl-}/\ce{Br-}/\ce{I-}). The viscoelastic and diffusion properties of the resulting materials are characterized by different rheological and calorimetric analyses, evidencing a strengthening of the gel network upon reinforcing the hard character of the cation and the soft character of the anionic counterion.

\section{Results and discussion}

\subsection{Synthesis and characterizations}

The phosphonate terminated polymer \ce{PBr}, which is the common basis to all studied polymer was synthesized according to our previously reported methodology; allowing us to obtain a well-controlled linear polymer ($Mw= \SI{5614}{\dalton}$, $Mw/Mn = 1.08$; S2, S3, Figure S1) .9, 10 All subsequent transformations of the polymer structures were achieved using the same initial batch of material. Nucleophilic addition of N-Methylimidazole or N-methylpyrrolidine to a heated solution of \ce{PBr} in THF afforded the corresponding poly(imidazolium) and poly(pyrrolidinium) compounds. In these cases, as an inherent consequence of the structure of the starting material, charge neutrality is provided by bromide counter ions, and the polymer will be referred to as \ce{PIm+Br-} and \ce{PPyr+Br-}, respectively (Scheme~\ref{sch:synthesis}).

\begin{scheme}
\includegraphics{synthesis.pdf}
\caption{Synthesis of \ce{PIm+Br-} and \ce{Ppyr+Br-} and their intermediate. $n=70$}
\label{sch:synthesis}
\end{scheme}

Anionic metathesis could be achieved by pouring an aqueous solution of \ce{PIm+Br-} or \ce{PPyr+Br-} into a saturated aqueous solution of the different sodium halides (\ce{NaF}, \ce{NaCl}, \ce{NaI}) (Scheme~\ref{sch:metathesis}). In the cases of \ce{PIm+F-}, \ce{PIm+Cl-} , \ce{PPyr+F-} and \ce{PPyr+Cl-}, a turbid suspension was obtained immediately after addition. After extensive dialysis of the resulting mixture against deionized water and lyophilization,, the different \ce{PIm+X-} and \ce{PPyr+X-} (where \ce{X}=\ce{F}, \ce{Cl}, \ce{I}) were recovered in high yields. In order to bring evidence for the efficiency of the ionic metathesis, the resulting materials were submitted to negative mode high resolution mass spectrometry (HRMS, S3). Displacement of bromide counterions could be unambiguously assessed by the full disappearance of the diisotopic mass peak (89/\SI{91}{\dalton}, Figures S8-9). Substitution by iodide was further confirmed by the concomitant apparition of a characteristic monoisotopic peak (\SI{121}{\dalton}, Figures S6-7), while fluoride and chloride anions signals, which are below the detection limits of the spectrometer, could not be observed in the corresponding polymers (see supporting information).

\begin{scheme}
\includegraphics{metathesis.pdf}
\caption{Anionic metathesis reaction. As above, \ce{Nu+} corresponds either to pyrrolidinium or imidazolium.}
\label{sch:metathesis}
\end{scheme}

Calorimetric data were then recorded in order to characterize the effect of counterion substitution. In solids, melting points are inherently connected to the reticular enthalpy of the crystal lattice, hence primarily to ion pairing strength.11 This last parameter is directly related to the diffuse and hindered nature of the interacting charges, bulkier and softer ions providing in the most general case weaker interactions.12 For typical ``ionic liquids'' such as imidazolium or pyrollidinium halides, melting points will thus gradually increase as the size of the halide counterion decreases, ie $\ce{I-}<\ce{Br-}<\ce{Cl-}<\ce{F-}$.13 Although the trend is less obvious in amorphous polymeric systems, in which phase transitions result from the interplay of multiple parameters, ionic substitution is however expected to have an impact on the melting temperature of the material.14

All polymers were thus studied by Differential Scanning Calorimetry (DSC). As expected, the lowest melting temperature was achieved for polymers with iodide counterions. Quite remarkably, a similar melting temperature (84-\SI{85}{\celsius}) was found for \ce{PIm+I-} and \ce{PPyr+I-}. When bromide counterions are present, a significant increase of the melting temperature was observed (114-\SI{115}{\celsius}) but again, a similar value was found for \ce{PIm+Br-} and \ce{PPyr+Br-}. In the case of fluoride and chloride, no phase transition was seen below decomposition temperature of the polymer (\SI{200}{\celsius}), which indicates a much stronger ion pairing in these materials (Figure~\ref{fig:dsc}, S4 Figures S2-5).

\begin{figure}
\caption{DSC analysis of a crystalline polymer samples showing the presence of melting point.}
\label{fig:dsc}
\end{figure}

\subsection{Gels preparation}
All materials were placed in vials, and water was added until a final aqueous dispersion of each polymer at a 8\%wt. concentration was obtained. Swelling properties turned out to be extremely dependent on the polymer compositions. With fluoride counterions, no observable swelling or dissolution took place and the suspension was recovered as a biphasic mixture of its individual components. Results were very similar for the chloride containg Poly(immidazolium) polymer, whereas a limited swelling was observed for its poly(pyrrolidinium) counterpart, resulting in a granular, inhomogeneous gel. When bromide was present, both polypyrrolidinium and polyimmidazolium polymers afforded a gel upon swelling with water. However, their appearance markedly differed, as the pyrrolidinium based system gave rise to a much harder and glassy-like material, indicating a consistently stronger interchain network. A similar gel behavior was found with \ce{PPyr+I-}, while \ce{PIm+I-}, although very similar in appearance to \ce{PIm+Br-} (homogeneous and optically transparent) turned out to be significantly harder when submitted to mechanical shearing (Figure~\ref{fig:pictures}).

\begin{figure}
\caption{pictures of 8\%wt water solutions of \ce{PIm+X-} and \ce{PPyr+X-} in water.}
\label{fig:pictures}
\end{figure}

Rheological measurements. In order to refine these visual observations, we performed rheological studies of the water-swelling materials. Briefly, data were recorded at \SI{25}{\degree} with an AR 1000 rheometer (TA Instruments) in a cone-plate geometry of radius \SI{40}{\milli\metre}, an angle of \SI{2}{\degree} and a truncation of \SI{58}{\micro\metre}.15, 16 The sample was placed on the plate, the cone was then lowered to the measuring position, spreading the sample in the process. Excess material was removed, so that the sample exactly filled the gap. Due to Weissenberg effect at high shear rate, no preshear was applied before measurement. To minimize water absorption, the geometry was covered with a solvent trap, using light mineral oil as a liquid seal between the rotor and the cap. After one minute of equilibration we performed an oscillatory frequency sweep at small amplitude (strain amplitude $\gamma=0.1\%$) and then an oscillatory strain sweep at fixed frequency ($f=\SI{1}{\hertz}$). For fluoride counterions as well as \ce{PIm+Cl-}, the rheological profile was that of pure water which confirms the visual observation. However as shown in Figure~\ref{fig:strain}, all other samples are solidlike ($G^\prime \gg G"$) at small strain and flow at large strains ($G" \gg G^\prime$). Solidlike behavior at low strain was confirmed for all accessible frequencies (see supplementary material S6 Fig. 1). We define the critical strain value $\gamma_c$ between these two regimes when $G"$ reaches a maximum. We checked that the linear mechanical properties are unchanged by the flow history given a few minutes of rest. %To understand the influence of the ion pairing on the mechanical properties, we compute the cohesive energy (equation 1) that should probe the strength of the physical supramolecular bonds.15,16

\begin{figure}
\caption{Storage modulus $G^\prime$ (filled symbols and solid lines) and loss modulus $G"$ (open symbols and dashed lines) measured through oscillatory shear experiments for (a) \ce{PPyr+Cl-} ($\bullet$), \ce{PPyr+Br-} ($\blacksquare$) and \ce{PPyr+I-} ($\blacktriangle$), (b) \ce{PIm+Br-} ($\blacklozenge$) and \ce{PPyr+Br-} ($\blacksquare$) and (c) \ce{PIm+I-} ($\blacktriangledown$) and \ce{PPyr+I-} ($\blacktriangle$). $G^\prime$  and $G"$ measured during an amplitude sweep at a fixed frequency $f=\SI{1}{\hertz}$ are plotted against the strain amplitude $\gamma$. All samples are at 8\%wt }
\label{fig:strain}
\end{figure}

In Figure~\ref{fig:strain}a we compare the rheological responses of Poly(pyrrolidinium) based hydrogels for the three counterions allowing dissolution in water. Gels with bromide and iodide counterions showed roughly the same mechanical behavior in the linear regime but \ce{PPyr+I-} yields at slightly higher strains ($\gamma_c=4\%$ instead of $3\%$). By contrast, the gel with chloride counterions is weaker by more than one order of magnitude. The cohesive energy goes down with smaller counterion indicating weaker intermolecular bonds (see supplementary material S6 table 1).

We also compare the rheology of \ce{PIm+X-} and \ce{PPyr+X-} with the same counterion \ce{X} in Figure~\ref{fig:strain}b,c. For both bromide (Figure~\ref{fig:strain}b, X=Br) and iodide (Figure~\ref{fig:strain}c, X=I), imidazolium based gels are much weaker than their pyrrolidinium counterpart.

\subsection{Discussion}

The above dependence of the gel parameters on the nature of the counterions could be rationalized on the basis of ion-pairing processes. The unsolubility/unswellability of \ce{PIm+F-}, \ce{PPyr+F-} and \ce{PIm+Cl-} suggest that ion pairs are not efficiently dissociated in aqueous medium, thus hindering the solvation process. By comparison, \ce{PIm+Br-}, \ce{PIm+I-}, \ce{PPyr+Br-} and \ce{PPyr+I-} form homogeneous materials when suspended in water, which one can correlate with efficient dissociation of the halide counterions from the polymer main chain is fully consistency with decreased ionic bond strength. Such dissociation, in turns makes it possible for the ``hard'' phosphonate group on each polymer chain's end to interact with the ``free'' cationic groups of the side chains, which we showed in a previous paper to be at the origin of the gel formation mechanism.9

The similarity in gel parameters observed for \ce{PPyr+Br-} and \ce{PPyr+I-} ($G^\prime\approx\SI{8500}{\pascal}$) suggests that full ion pair dissociation is achieved in the latter, or at least that this dissociation is enough in both cases to provide maximal interaction between the terminal phosphonate and the pyrrolidinium side groups. Conversely the significantly different behavior between \ce{PIm+I-} ($G^\prime= \SI{78}{\pascal}$) and \ce{PIm+Br-} ($G^\prime= \SI{13}{\pascal}$) indicates that dissociation is only partial and occurs to a lesser extent in the latter than in the former. The same conclusion can be drawn for \ce{PPyr+Cl-}, in which the elastic modulus ($G^\prime= \SI{270}{\pascal}$) value drops by more than one order of magnitude as compared to its bromide and iodide counterparts.

As already mentioned (vide supra) another striking feature of these gels is that in similar conditions poly(pyrrolidinium) based gels are systematically much stronger than those formed with poly(imidazolium). Indeed, although both cationic in nature, pyrrolidinium and imidazolium ions have markedly different properties: whereas pyrrolidinium features a ``hard'' positive charge very localized on the quaternary nitrogen, imidazolium has a much more delocalized ``softer'' charge; consequently, according to the Hard and Soft Acid and Bases (HSAB) theory, stronger association is expected for the latter with the ``soft'' halide counterions.17 As a consequence, for a given halide, dissociation is less favored in the case of poly(imidazolium), as clearly illustrated by the difference in solvation between \ce{PIm+Cl-} and \ce{PPyr+Cl-}. Conversely, association of the lateral cations with the ``hard'' phosphonate moieties at the polymer chain’s end is expected to be stronger with pyrrolidinium, hence a systematically larger elastic modulus. The interplay of these two effects contributes in making poly(pyrrolidinium) based gels much stronger than their poly(imidazolium) counterparts; comparison between \ce{PIm+Br-} and \ce{PPyr+Br-} is particularly illustrative of this trend, as their $G^\prime$ value (13 and \SI{8100}{\pascal}, resp.) differ by almost three orders of magnitude.

\section{Conclusion}
With this work we demonstrated the versatility of our ``head-to-body'' electrostatic approach in the fabrication of hydrogels with readily tunable rheological properties. Thus, depending on the nature of the nucleophile used in the quaternarization step of the polymer, but also on that of the counterion which can be modified in the course of its purification process, we showed that it is possible to manipulate almost at will the gel formability and respective magnitudes of the shear modulus and of the shear stress. We don’t observe the formation of a shear thinning gel when the halide counterion is ``hard.'' Consistent, the strongest gels are obtained when a combination of ``hard'' cations and ``soft'' halides are used, which gives rise, in line with the HSAB theory, to maximal charge dissociation in water: this in turn facilitates charge pairing of the peripheral, iterative, cations with the terminal anion. It is thus possible to obtain robust hydrogels with short, linear polymer chains and in the absence of any additive, which could find interesting applications especially in the context of biomaterials.

\begin{acknowledgement}
The authors are grateful for the financial support of Université de Lyon through the program ``Investissements d'Avenir'' ( ANR-1 1-IDEX-0007). S.M. and N.T. acknowledge funding from the European Research Council under the European Union's Seventh Framework Program (FP7/2007–2013)/ERC grant agreement No.258803.
\end{acknowledgement}

\begin{suppinfo}

Materials and methods, detailed synthesis and characterizations of gels, GPC, HRMS and DSC data, calculated cohesion energy, and frequency sweep curves.

\end{suppinfo}


\end{document}