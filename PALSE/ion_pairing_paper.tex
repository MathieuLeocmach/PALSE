\documentclass[journal=jacsat,manuscript=article]{achemso}
\usepackage[utf8]{inputenc}
\usepackage[T1]{fontenc}       % Use modern font encodings
\usepackage{amssymb}
\usepackage[version=3]{mhchem} % Formula subscripts using \ce{}
\usepackage{siunitx}
\usepackage[table]{xcolor}



\author{Hassan Srour}
\affiliation[Laboratoire de Chimie de l'ENS de Lyon]{Laboratoire de Chimie UMR CNRS 5182 Ecole Normale Supérieure de Lyon/ Université Claude Bernard Lyon1/ Université de Lyon 46 Allée d'Italie, 69007 Lyon}


\author{Mathieu Leocmach}
\affiliation[Institut Lumière Matière]{Institut Lumière Matière, CNRS UMR 5306, Université Claude Bernard Lyon 1, Université de Lyon, Lyon, 69622 Villeurbanne Cedex, France}
\email{mathieu.leocmach@univ-lyon1.fr}

\author{Martien Duvall Deffo Ayagou}
\author{Thi Thanh-Tam Nguyen}
\affiliation[Laboratoire de Chimie de l'ENS de Lyon]{Laboratoire de Chimie UMR CNRS 5182 Ecole Normale Supérieure de Lyon/ Université Claude Bernard Lyon1/ Université de Lyon 46 Allée d'Italie, 69007 Lyon}

\author{Nicolas Taberlet}
\author{Sébastien Manneville}
\affiliation[Laboratoire de Physique de l'ENS de Lyon]{Laboratoire de Physique, Ecole Normale Supérieure de Lyon/ Université Claude Bernard Lyon1/ Université de Lyon, 46 Allée d'Italie, 69007 Lyon}


\author{Chantal Andraud}
\author{Cyrille Monnereau}
\affiliation[Laboratoire de Chimie de l'ENS de Lyon]{Laboratoire de Chimie UMR CNRS 5182 Ecole Normale Supérieure de Lyon/ Université Claude Bernard Lyon1/ Université de Lyon 46 Allée d'Italie, 69007 Lyon}
\email{cyrille.monnereau@ens-lyon.fr}

%\title{Supramolecular crocodile line controls the rheological properties of polyelectrolytes hydrogels}
\title{Ion pairing controls rheological properties of ``crocodile-line'' polyelectrolyte hydrogels}
%\title{Ion pairing in crocodile line hydrogels controls rheological properties}
%\title{In crocodile line hydrogels ion pairing controls rheological properties}
%\title{Controlling the rheological properties of well-defined polyelectrolytes hydrogels by fine tuning their ion pairing process}


\keywords{ATRP, hydrogel, poly(cations), rheology}

\begin{document}

\begin{tocentry}

9\,cm by 3.5\,cm

\end{tocentry}

\begin{abstract}
We demonstrated recently that polyelectrolytes with cationic moieties along the chain and a single anionic head are able to form physical hydrogels due to the reversible nature of the head-to-body ionic bond. Here we generate a variety of such polyelectrolytes with various cationic moieties and counter-ions combinations starting from a common polymeric platform. We show that the rheological properties (shear modulus, critical strain) of the final hydrogels vary by three orders of magnitude depending on the cation/anion pair. Our data are well described by a theoretical model involving a supramolecular crocodile line between cross-links, the length of the line being set by the amount of charge dissociation.
\end{abstract}

\section{Introduction}
Since the pioneering work of \citet{Wichterle1960} which established their relevance in a biomedical context, polymer-based hydrogels have never ceased to be a very active field of research\cite{Calo2015,Hoffman2001}. They have recently experienced a burst of interest among the biomedical community as controlled drug delivery cargos or scaffolds for wound healing or tissue repair\cite{Vashist2014,Ratner2004}. Polyelectrolytes are being extensively put to use in this particular context\cite{Rosso2003,BinImran2014}. Their gel formation mechanism often involves reversible-by-nature electrostatic interactions, which can be used for instance to trigger ``smart'' release of bioactive substances\cite{Khare1993,Lockwood2007}. Besides, poly(cationic) gels have been reported to combine scaffold properties for cell’s adherence and growth with antimicrobial activity, and are therefore increasingly used in clinical applications\cite{Hoffman2001,Landers2002}. In this framework, injectability is a key feature, as it provides an easier way to gel delivery in vivo\cite{Tibbitt2016}; thus, reversible shear thinning biocompatible hydrogels are a particularly sought after class of materials\cite{Yu2008}.

We have recently reported on a highly reversible poly(cation) based hydrogel\cite{Srour2014}, which formation relies on an new concept of supramolecular  electrostatic interaction. Briefly, the atom transfer radical polymerisation (ATRP) is initiated by a phosphonate-terminated ATRP initiator, eventually affording an anion-terminated poly(cationic) polymer. The resulting poly(cationic) polymer, when dispersed in an aqueous medium, stabilizes hydrogel formation through the occurrence of a head(anion)-to-body(cations) supramolecular network. This highly dynamic electrostatic network provides the resulting gel with spectacular mechanical and self-healing properties. Moreover the interaction can be disrupted by various chemical stimuli, such as pH or ionic strength.

In the present report, we investigate the role of the dissociation rate of the poly(cationic) body on the mechanical properties of the hydrogel. In a first part we describe how we take advantage of our well-controlled synthesis to play systematically with the nature (hardness/softness) of the cationic repeating unit and the ion-paring strength of the associated counter-ion (\ce{F-}/\ce{Cl-}/\ce{Br-}/\ce{I-}). In a second part, we show the qualitative and quantitative variations of mechanical properties of the aqueous dispersions obtained from these well-characterized polymers. In a third part, we develop a microscopic model based on the idea that at low dissociation rates, cross-links are not separated by a single macromolecule but by several (up to hundreds) polymers in a supramolecular crocodile line. %Finally, we discuss our results and possible applications of such smart hydrogels.

\section{Results and discussion}

\subsection{Synthesis and characterizations}

We synthesise the phosphonate terminated polymer hereafter reffered to as \ce{PBr} according to our previously reported methodology\cite{Srour2014,Appukuttan2012}. We obtain a well-controlled linear polymer ($Mw= \SI{5614}{\dalton}$, $Mw/Mn = 1.08$; see Supplementary Materials S2, S3, Figure S1), of polymerisation index $N_0=70$, which is the common basis to all studied polymers. We achieve all subsequent transformations of the polymer structures using the same initial batch of material. Nucleophilic addition of N-methylimidazole or N-methylpyrrolidine to a heated solution of \ce{PBr} in THF affords the corresponding poly(imidazolium) and poly(pyrrolidinium) compounds. In these cases, as an inherent consequence of the structure of the starting material, charge neutrality is provided by bromide counter ions, and the polymer will be referred to as \ce{PIm+Br-} and \ce{PPyr+Br-}, respectively (Scheme~\ref{sch:synthesis}).

\begin{scheme}
\includegraphics{synthesis.pdf}
\caption{Synthesis of \ce{PIm+Br-} and \ce{PPyr+Br-} and their intermediates \ce{POH} and \ce{PBr} with $N_0=70$.}
\label{sch:synthesis}
\end{scheme}

We performed anionic metathesis by pouring an aqueous solution of \ce{PIm+Br-} or \ce{PPyr+Br-} into a saturated aqueous solution of the different sodium halides (\ce{NaF}, \ce{NaCl}, \ce{NaI}) (Scheme~\ref{sch:metathesis}). In the cases of \ce{PIm+F-}, \ce{PIm+Cl-} , \ce{PPyr+F-} and \ce{PPyr+Cl-}, we obtain a turbid suspension immediately after addition. After extensive dialysis of the resulting mixture against deionized water and lyophilization we recover in high yields the different \ce{PIm+X-} and \ce{PPyr+X-} (where \ce{X}=\ce{F}, \ce{Cl}, \ce{I}). In order to bring evidence for the efficiency of the ionic metathesis, we submit the resulting materials to negative-mode high-resolution mass spectrometry (HRMS, see Supplementary Materials S3). We unambiguously assess displacement of bromide counterions by the full disappearance of the diisotopic mass peak (89/\SI{91}{\dalton}, Figures S8-9). We further confirm substitution by iodide by the concomitant apparition of a characteristic monoisotopic peak (\SI{121}{\dalton}, Supplementary Figures S6-7), while fluoride and chloride anions signals, which are below the detection limits of the spectrometer, could not be observed in the corresponding polymers (see supporting information XXX S? XXX).

\begin{scheme}
\includegraphics{metathesis.pdf}
\caption{(a) Anionic metathesis reaction. As above, \ce{Nu+} corresponds either to pyrrolidinium or imidazolium. (b) Summary of the polymers derived from a single batch of \ce{PBr}.}
\label{sch:metathesis}
\end{scheme}

In ionic solids, melting points are inherently connected to the reticular enthalpy of the crystal lattice, hence primarily to ion pairing strength\cite{Sherman1932}. This last parameter is directly related to the diffuse and hindered nature of the interacting charges, bulkier and softer ions providing in the most general case weaker interactions\cite{Krossing2006}. For typical ``ionic liquids'' such as imidazolium or pyrrolidinium halides, melting points will thus gradually increase as the size of the halide counterion decreases, ie $\ce{I-}<\ce{Br-}<\ce{Cl-}<\ce{F-}$\cite{Dean2010}. Although analysis is less straightforward in the case of amorphous polymer electrolytes, ionic substitution is however expected to have an impact on the phase transition temperatures of the material\cite{Orler1994}.

In order to characterize the effect of counterion substitution, we study all polymers by Differential Scanning Calorimetry (DSC). Although we observe no glass transition in the studied temperature range for any of the sample, comparison of the DSC data reveal significant differences.

In the case polyelectrolytes with fluoride and chloride counterion, we observe no exo- or endothermic transition below \SI{200}{\celsius}. By contrast, we observe a broad endothermic peak when iodide and bromide are used as counterions. Quite remarkably, we found a similar peak temperature (84-\SI{85}{\celsius}) for \ce{PIm+I-} and \ce{PPyr+I-}, see Figure~\ref{fig:dsc}. When bromide counterions are present, we observe a significant increase of the peak temperature (114-\SI{115}{\celsius}) but again, with a similar value between \ce{PIm+Br-} and \ce{PPyr+Br-} as shown in Figure~\ref{fig:dsc}. The overall shape and position of the DSC peaks is very reminiscent of previously reported data for various naturally occuring or synthetic poly(electrolytes)\cite{Li2005,Sarmento2006,Ostrowska-Czubenko2009a,Moin2015}. It is generally attributed to desorption of weakly bound water from the polymer network. Its presence in the polymer samples where bromide and iodide are used as couterions suggests that adsorbed water could play a role in solvating the dissociated ion pairs in the latter. Conversely, its absence for chloride and fluoride counterions suggests that dissociation is not effective because of stronger ion pair interaction.

\begin{figure}
\includegraphics{dsc.pdf}
\caption{DSC measurements of polymer samples showing the presence of an endothermic peak. Curves are shifted vertically by arbitrary amounts for clarity. Vertical lines indicate the peak temperatures at \SI{84}{\celsius} for the two iodides in black and \SI{114}{\celsius} for the two bromides in gray.}
\label{fig:dsc}
\end{figure}

\subsection{Water swelling, ion-pairing and solvent quality}
Water-swelling properties turn out to be extremely dependent on the polymer composition. With bromide or iodide conterions, both poly(pyrrolidinium) and poly(immidazolium) polymers afford homogeneous and optically transparent gels upon swelling with deionised water. We observe a limited swelling for \ce{PPyr+Cl-}, resulting in a granular, inhomogeneous gel. With fluoride counterions, or in the case of \ce{PIm+Cl-}, we observe neither swelling nor dissolution. We recover a 8\%wt suspension as a biphasic mixture of its individual components.

The above dependence of gel formability on the nature of the counterions could be rationalized on the basis of ion-pairing processes. According to the Hard and Soft Acid and Bases (HSAB) theory, stronger association is expected between a ``softer'' cationic moiety and a ``harder'' halide counterions\cite{Goossens2009}. Therefore charge dissociation in water and thus solubility of the polymer should be easier with larger halide and pyrrolidinium. Indeed, although both cationic in nature, pyrrolidinium and imidazolium ions have markedly different properties: whereas pyrrolidinium features a ``hard'' positive charge very localized on the quaternary nitrogen, imidazolium has a much more delocalized ``softer'' charge. That is why we are able to dissolve the five polymers that have the weakest ion pairing, but neither the two fluorides nor \ce{PIm+Cl-} where counterions are expected to be strongly bound to the chain.

A fortiori, the insolubility in absence of charge dissociation indicates that the chain is in poor solvent. To estimate the $\Theta$-temperature, we held the two fluoride polymers in water at boiling temperature overnight without any dissolution, implying that $\Theta>\SI{100}{\celsius}$. Nonetheless, due to charge dissociation, five of our polyelectrolytes are soluble at \SI{25}{\celsius}.


%\begin{figure}
%\caption{pictures of 8\%wt water solutions of \ce{PIm+X-} and \ce{PPyr+X-} in water.}
%\label{fig:pictures}
%\end{figure}

\subsection{Rheological measurements}
We perform rheological studies of the water-swelling materials. Briefly, data are recorded at \SI{25}{\degree} with an AR 1000 rheometer (TA Instruments) in a cone-plate geometry of radius \SI{40}{\milli\metre}, an angle of \SI{2}{\degree} and a truncation of \SI{58}{\micro\metre}\cite{Macosko1994,Larson1999}. We place the sample on the plate, and then lower the cone to the measuring position, spreading the sample in the process. We remove excess material, so that the sample exactly fills the gap. Due to Weissenberg effect at high shear rate, we apply no preshear before measurement. To minimize water absorption, we cover the geometry with a solvent trap, using light mineral oil as a liquid seal between the rotor and the cap. After one minute of equilibration we perform an oscillatory frequency sweep at small amplitude (strain amplitude $\gamma=0.1\%$) and then an oscillatory strain sweep at fixed frequency ($f=\SI{1}{\hertz}$). For fluoride counterions as well as \ce{PIm+Cl-}, the rheological profile is that of pure water which confirms the visual observation. However as shown in Figure~\ref{fig:strain}, all other samples are solidlike ($G^\prime \gg G^{\prime\prime}$) at small strain and flow at large strains ($G^{\prime\prime} \gg G^\prime$). We confirm solidlike behavior at low strain for all accessible frequencies (see supplementary material S6 Fig. 1). We define the critical strain value $\gamma_c$ between these two regimes when $G^{\prime\prime}$ reaches a maximum. We checked that the linear mechanical properties are unchanged by the flow history given a few minutes of rest. %To understand the influence of the ion pairing on the mechanical properties, we compute the cohesive energy (equation 1) that should probe the strength of the physical supramolecular bonds.15,16

\begin{figure}
\includegraphics{strain.pdf}
\caption{Storage modulus $G^\prime$ (filled symbols) and loss modulus $G^{\prime\prime}$ (open symbols) measured through oscillatory shear experiments plotted against the strain amplitude $\gamma$ for (a) \ce{PPyr+Cl-} (\textcolor{lightgray}{$\bullet$}), \ce{PPyr+Br-} (\textcolor{gray}{$\blacksquare$}) and \ce{PPyr+I-} ($\blacktriangle$), (b) \ce{PIm+I-} ($\blacktriangledown$) and \ce{PPyr+I-} ($\blacktriangle$) and (c) \ce{PIm+Br-} (\textcolor{gray}{$\blacklozenge$}) and \ce{PPyr+Br-} (\textcolor{gray}{$\blacksquare$}). The fixed frequency is $f=\SI{1}{\hertz}$. All samples are at 8\%wt }
\label{fig:strain}
\end{figure}

In Figure~\ref{fig:strain}a we compare the rheological responses of Poly(pyrrolidinium)-based hydrogels for the three counterions allowing dissolution in water. Gels with bromide and iodide counterions show roughly the same mechanical behavior in the linear regime but \ce{PPyr+I-} yields at slightly higher strains ($\gamma_c=4\%$ instead of $3\%$). By contrast, the gel with chloride counterions is weaker by more than one order of magnitude. In other words, a ``soft'' halide counterion, efficiently dissociated, is correlated to a harder gel.

We also compare the rheology of \ce{PIm+X-} and \ce{PPyr+X-} with the same counterion \ce{X} in Figure~\ref{fig:strain}b,c. For both iodide (Figure~\ref{fig:strain}b, X=I) and bromide (Figure~\ref{fig:strain}c, X=Br), imidazolium based gels are much weaker than their pyrrolidinium counterpart with a broader linear regime. Comparison between \ce{PIm+Br-} and \ce{PPyr+Br-} is particularly illustrative of this trend, as their $G^\prime$ value (13 and \SI{8100}{\pascal}, resp.) and $\gamma_c$ value (804\% and 1.6\%, resp.) differ by almost three orders of magnitude. Here also, gel hardness is correlated to efficient charge dissociation, this time via ``harder'' cationic moieties. In the following we explain quantitatively this correlation from a microscopic model.



\subsection{Crocodile line model}

\begin{figure}
\includegraphics{dissociation.pdf}
\caption{Crocodile line model. (a) Sketch of the network in the case of a crocodile line of size $n=3$ between cross-links (gray disks) and a persistence length $r_\mathrm{src}\approx 3D$. Empty circles are not individual monomers but electrostatic blobs of size $D$. Electrostatic blobs containing an anionic head are shown as black filled circles. (b) Relation between the number of monomers between uncondensed charges and the number of chains between cross-link points. The line corresponds to $n = A^2$.}
\label{fig:dissociation}
\end{figure}

We have previously demonstrated that, due to the opposite charges of the phosphonate head, head-to-body ionic bonds are possible~\cite{Srour2014}. On the one hand, if at least two foreign heads attach to the same body, we obtain an effective cross-link point. When the probability of such a configuration is non-zero, we obtain a physically cross-linked gel.

On the other hand, if every chain has a single foreign head attached to its body, every polymer is linked to two others in a crocodile line\bibnote{We neglect the possibility of the two charges of the phosphonate head to attach to two separate chains}. In addition, if we suppose that in order to minimize the inter-chain repulsion between charged cationic groups, heads are preferentially attached to the tail of their neighbour, we obtain a linear chain of effective polymerisation index $n N_0$ where $n$ is the number of polymers in the crocodile line.

We thus have two limiting cases: (i) every chain has at least two heads attached and we have roughly $N_0$ monomers between cross-link points; (ii) every chain has a single head attached and we have a single supramolecular chain in the system. The former case should be observed at high charge dissociation when many moieties are ionised along the chain providing many sites for a phosphonate head to bound to. The latter case should be observed at negligible charge dissociation. In between these two limiting cases, we should observe cross-link points separated by crocodile lines of $n N_0$ monomers as sketched in Figure~\ref{fig:dissociation}a.

Let us now discuss the rheological measurements and link them to the hydrogel microstructure. We reduce our measurements to two main observables, see Table~\ref{tab:results}: (i) the value of the shear modulus at small amplitude $G^\prime(\gamma\rightarrow 0)$, i.e. the elasticity of the undamaged gel network, and (ii) the critical strain $\gamma_c$ corresponding to the maximum of $G"$, that we consider as representative of the end of the linear regime and of the first broken cross-links. XXX Ref XXX


\subsubsection{From modulus to effective chain length}

Between two cross-links the crocodile chain can be considered as an entropic spring made of $N=n N_0$ monomers. The shear modulus of the gel is then given by\cite{Rubinstein1996}:
\begin{equation}
G = \frac{c}{N}k_\mathrm{B}T,
\label{eq:G}
\end{equation}
where $c$ is the monomer number density, $k_\mathrm{B}$ is the Boltzmann constant and $T$ the temperature. Knowing the molecular mass $M$ of the chains, density $d$ of the solvent and the weight fraction $w$ of polymer, we obtain
\begin{equation}
n = \frac{\mathcal{N_A}}{M} w d \frac{k_\mathrm{B}T}{G}.
\end{equation}

We find that the number $n$ of chains between cross-link point goes from 1 in \ce{PPyr+I-} and \ce{PPy+Br-} to 790 in \ce{PIm+Br-}, following the \textit{a priori} ranking of charge dissociation, see Table~\ref{tab:results}. This is consistent with a higher probability of attaching two or more heads on a highly dissociated body, i.e. a large cross-link ratio.

\begin{table}
\begin{tabular}{l|SS|SSSSS}
& {$G^\prime(\gamma\rightarrow 0)$} & {$\gamma_c$} & {$n$} & \multicolumn{2}{c}{$E_c$} & {$A$} & {$\sqrt{n}$}\\
&	{\si{\pascal}} &  & & {$k_\mathrm{B}T$} & {\si{\kilo\joule/\mol}} & \\\hline&&&&&\\[-10pt]
\ce{PPyr+I-}	& 8800	&	0.04	&	1.	&	2e-3\cellcolor{gray!25}	&	0.004\cellcolor{gray!25}	&	1.	&	1.\\
\ce{PPyr+Br-}	& 9800 	& 	0.016	&	1.05&	3e-4\cellcolor{gray!25}	&	0.001\cellcolor{gray!25}	&	1.1	&	1.\\
\ce{PPyr+Cl-}	& 270 	&	0.09	&	45	&	8e-3\cellcolor{gray!25}	&	0.02\cellcolor{gray!25}	&	7.	&	6.7\\
\ce{PIm+I-}	& 78	&	0.16	&	113	&	3e-2\cellcolor{gray!25}	&	0.063\cellcolor{gray!25}	&	9.6	&	10.6\\
\ce{PIm+Br-}	& 13	&	8.04	&	790	&	64	&	160.	&	3.4\cellcolor{gray!25}	&	28\\
\end{tabular}
\caption{Summary of rheological measurements and microscopic values deduced from the model. Gray background indicates inconsistent values obtained either by (Eq.~\ref{eq:Ec}) or by (Eq.~\ref{eq:A}) outside of their respective validity domain.}
\label{tab:results}
\end{table}

\subsubsection{Bounding energy}

Considering that each crocodile chain is an entropic spring in its linear domain, the mechanical energy stored by a single crocodile chain under a strain $\gamma$ is $E = \gamma^2 k_\mathrm{B}T$. If we suppose that ionic bonds break before the complete extension of the crocodile chains, we evaluate the head-to-body bonding energy to
\begin{equation}
E_c = \gamma_c^2 k_\mathrm{B}T.
\label{eq:Ec}
\end{equation}

For \ce{PIm+Br-} we measure $\gamma_c = 800\%$ and thus $E_c \approx 65 k_\mathrm{B}T$ per head-to-body bond or \SI{160}{\kilo\joule\per\mol}. This value is much higher than the $E_i\approx\SI{5}{\kilo\joule\per\mol}$ typical of ionic bonds in water\cite{Schneider1992}, even when adding up the two bonds of the bivalent phosphonate head\bibnote{Since the two minus charges are born by the same atom, chelation effect may enhance bounding energy, but not by an order of magnitude}. However, we have to remember that the chain is in bad solvent, thus forming locally compact blobs almost devoid of water. Therefore, the medium surrounding the ionic bond is not water but the hydrophobic chains. Such an apolar microenvironement is well known to enhance otherwise weak electrostatic interactions in protein folding or engineered self-assembly\cite{Rehm2010}. Quantitatively, the bonding energy of an ionic bond is inversely proportional to the relative dielectric constant $\epsilon_r$, therefore we need to suppose a local $\epsilon_r$ around 5 (instead of 80 in water) to recover the large bonding energy we measure.

However for all other samples the critical strain is between 2\% and 20\%, leading to inconsistent values for the bonding energies ranging between $10^{-3}k_\mathrm{B}T$ and $10^{-1}k_\mathrm{B}T$ per bond, see Table~\ref{tab:results}. Therefore, the end of the linear regime cannot correspond to bond breaking but rather to the point where extended chains become nonlinear.


\subsubsection{From critical strain to charge dissociation}

The spatial conformation of a polyelectrolyte chain in semi-dilute, low salt solution is a persistent random walk where the persistence length is the screening length $r_\mathrm{src}$ due to the dissociated counter ions\cite{Rubinstein1996}. In bad solvent, the persistent random walk is made of electrostatic blobs of diameter $D$ containing $g_e$ monomers, see Figure~\ref{fig:dissociation}a:
\begin{align}
g_e &= \frac{A^2}{u}\tau\label{eq:ge},\\
D &= b \left(\frac{A^2}{u}\right)^{1/3}\label{eq:D}.
\end{align}
Here $A$ is the number of monomers between dissociated ion pairs, $\tau = 1 - T/\Theta$ the reduced temperature and $u = \ell_\mathrm{B}/b \approx 2.7$ is the ratio of the Bjerrum length in water $\ell_\mathrm{B}\approx \SI{0.7}{\nano\metre}$ and monomer size $b\approx \SI{0.26}{\nano\metre}$. 

Extending these electrostatic blobs means exposing more monomers to the bad solvent and typically requires energies at least an order of magnitude higher than for the extension of the persistent random walk. We will thus consider that the exit from the linear domain of the material corresponds to the full extension of the persistent random walk:
\begin{equation}
\gamma_c = \frac{R_0}{R} - 1
\end{equation}
where $R$ and $R_0$ are respectively the end-to-end distance and the contour length of the persistent random walk.

A segment of the persistent random walk is $r_\mathrm{scr}$ long and thus contains $g_\mathrm{scr} = g_e r_\mathrm{scr}/D$ monomers. We thus have
\begin{equation}
\left. \begin{array}{ll}
R &= r_\mathrm{scr} \left(\frac{N}{g_\mathrm{scr}}\right)^{1/2}\\
R_0 &= r_\mathrm{scr} \frac{N}{g_\mathrm{scr}}
\end{array}\right\rbrace\quad
\gamma_c = \left(\frac{N}{g_\mathrm{scr}}\right)^{1/2} -1\label{eq:gamma0}.
\end{equation}

\citeauthor{Rubinstein1996} express the screening length as
$r_\mathrm{scr} = \left(A^2 / u\right)^{1/3} \left(\tau /cb \right)^{1/2}$. Combining with (Eq.~\ref{eq:ge}) and (Eq.~\ref{eq:D}) we deduce
\begin{equation}
g_\mathrm{scr} = \frac{A^2}{u} \left(\frac{\tau^3}{c b^3}\right)^{1/2}\label{eq:gscr}.
\end{equation}

From (Eq.~\ref{eq:gamma0}) and (Eq.~\ref{eq:gscr}) we can finally relate the measured critical strain to the microscopic parameter $A$:
\begin{equation}
A = \frac{\left(N u\right)^{1/2}}{\gamma_c+1}\left(\frac{c b^3}{\tau^3}\right)^{1/4}.
\label{eq:A}
\end{equation}

The only unknown here is the reduced temperature $\tau$, possibly different between poly(pyrrolidinium) based and poly(immidazolium) based polymers. We know that the undissociated chain has a $\Theta$ temperature higher than \SI{100}{\celsius}, thus $0.2<\tau<1$. In addition, we must have $A\geq 1$, which implies through (Eq.~\ref{eq:A}) $\tau\leq 0.235$, corresponding to $\Theta\approx\SI{116}{\celsius}$. This limit value yields $A=1$ for both \ce{PPyr+I-} and \ce{PPyr+Br-}, indicating full dissociation of the cationic moieties. By contrast \ce{PPyr+Cl-} has a charge every 7 monomers. Supposing the same $\tau$, \ce{PIm+I-} has a dissociated ion pair every 10 monomers, see Table~\ref{tab:results}. Of course analysing \ce{PIm+Br-} based on the value of $\gamma_c$ gives an inconsistent value $A=3.4$ indicating that the hydrogel doesn't belong to the same regime.

\subsubsection{Charge dissociation and cross-linking ratio}

Finally we cross-check our model by linking the ion pair dissociation and the amount of physical cross-links by an independent argument. If we assume that the probability for an ion pair to dissociate is independent of the number of dissociated pairs already present on the polymer, we can describe the probability of having $k$ dissociated pairs on a polymer by a Poisson distribution of mean $\lambda = N_0/A$. Having 2 charges dissociated on a chain implies a cross-link. Therefore, the cross-link probability $1/n$ is proportional to $\lambda^2$, yielding $n \sim A^2$.

As shown in Figure~\ref{fig:dissociation}b and the last two columns of Table~\ref{tab:results} this relation describes very well our data. We can extrapolate this law for the case of \ce{PIm+Br-} and obtain $A = 28$. Inverting (Eq.~\ref{eq:A}) yields $R_0/R - 1 = 860\%$, indeed higher than the observed $\gamma_c\approx 804\%$. This confirms that in the \ce{PIm+Br-} strain sweep $\gamma_c$ actually results from the strength of the ionic bonds.

%We can now estimate the hydrophobic interaction energy $E_h$ by considering that to separate two chains two interfaces with water of area $\pi D^2/4$ have to be created. Surface tension is $\tau^2 k_\mathrm{B}T / b^2$\cite{Rubinstein1996}. Combining with (Eq.~\ref{eq:D}) we find
%\begin{equation}
%\frac{E_i}{k_\mathrm{B}T} = \frac{\pi}{2}\tau^2\left(\frac{A^2}{u}\right)^{2/3} \approx 4
%\end{equation}






%\subsection{Discussion}
%
%The above dependence of the gel parameters on the nature of the counterions could be rationalized on the basis of ion-pairing processes. The unsolubility/unswellability of \ce{PIm+F-}, \ce{PPyr+F-} and \ce{PIm+Cl-} suggest that ion pairs are not efficiently dissociated in aqueous medium, thus hindering the solvation process. By comparison, \ce{PIm+Br-}, \ce{PIm+I-}, \ce{PPyr+Br-} and \ce{PPyr+I-} form homogeneous materials when suspended in water, which one can correlate with efficient dissociation of the halide counterions from the polymer main chain is fully consistency with decreased ionic bond strength. Such dissociation, in turns makes it possible for the ``hard'' phosphonate group on each polymer chain's end to interact with the ``free'' cationic groups of the side chains, which we showed in a previous paper to be at the origin of the gel formation mechanism\cite{Srour2014}.
%
%The similarity in gel parameters observed for \ce{PPyr+Br-} and \ce{PPyr+I-} ($G^\prime\approx\SI{8500}{\pascal}$) suggests that full ion pair dissociation is achieved in the latter, or at least that this dissociation is enough in both cases to provide maximal interaction between the terminal phosphonate and the pyrrolidinium side groups. Conversely the significantly different behavior between \ce{PIm+I-} ($G^\prime= \SI{78}{\pascal}$) and \ce{PIm+Br-} ($G^\prime= \SI{13}{\pascal}$) indicates that dissociation is only partial and occurs to a lesser extent in the latter than in the former. The same conclusion can be drawn for \ce{PPyr+Cl-}, in which the elastic modulus ($G^\prime= \SI{270}{\pascal}$) value drops by more than one order of magnitude as compared to its bromide and iodide counterparts.
%
%As already mentioned (vide supra) another striking feature of these gels is that in similar conditions poly(pyrrolidinium) based gels are systematically much stronger than those formed with poly(imidazolium). Indeed, although both cationic in nature, pyrrolidinium and imidazolium ions have markedly different properties: whereas pyrrolidinium features a ``hard'' positive charge very localized on the quaternary nitrogen, imidazolium has a much more delocalized ``softer'' charge; consequently, according to the Hard and Soft Acid and Bases (HSAB) theory, stronger association is expected for the latter with the ``soft'' halide counterions\cite{Goossens2009}. As a consequence, for a given halide, dissociation is less favored in the case of poly(imidazolium), as clearly illustrated by the difference in solvation between \ce{PIm+Cl-} and \ce{PPyr+Cl-}. Conversely, association of the lateral cations with the ``hard'' phosphonate moieties at the polymer chain’s end is expected to be stronger with pyrrolidinium, hence a systematically larger elastic modulus. The interplay of these two effects contributes in making poly(pyrrolidinium) based gels much stronger than their poly(imidazolium) counterparts; comparison between \ce{PIm+Br-} and \ce{PPyr+Br-} is particularly illustrative of this trend, as their $G^\prime$ value (13 and \SI{8100}{\pascal}, resp.) differ by almost three orders of magnitude.

\section{Conclusion}
With this work we demonstrated the versatility of our ``head-to-body'' electrostatic approach in the fabrication of hydrogels with readily tunable rheological properties. Thus, depending on the nature of the nucleophile used in the quaternarization step of the polymer, but also on that of the counterion which can be modified in the course of its purification process, we showed that it is possible to manipulate almost at will the gel formability, the density of physical cross-links and the respective magnitudes of the shear modulus and of the shear stress. 

When the halide counterion is ``hard'', there is not enough charge dissociation to allow dissolution and gel formation. On the other hand of the spectrum outlined by HSAB theory, the hardest gels with the narrowest linear domain are obtained when a combination of ``hard'' cations and ``soft'' halides are used. This gives rise to maximal charge dissociation in water, facilitates charge pairing of the peripheral, iterative cations with the terminal anion and thus affords a very high density of cross-links. In between, in the case of ``hard'' cations and not-so-hard halide (\ce{Cl-}) or in the ``soft-soft'' case, charge dissociation is limited, polymers associate in crocodile lines and cross-links are few. We thus obtain soft, highly deformable gels.

To conclude, our procedure yields robust, highly tunable hydrogels from short, linear polymer chains and in the absence of any additive, which could find interesting applications especially in the context of biomaterials.

\begin{acknowledgement}
The authors are grateful for the financial support of Université de Lyon through the program ``Investissements d'Avenir'' (ANR-1 1-IDEX-0007). S.M. and N.T. acknowledge funding from the European Research Council under the European Union's Seventh Framework Program (FP7/2007–2013)/ERC grant agreement No.258803.
\end{acknowledgement}

\begin{suppinfo}

Materials and methods, detailed synthesis and characterizations of gels, GPC, HRMS and DSC data, calculated cohesion energy, and frequency sweep curves.

\end{suppinfo}

\bibliography{PALSE.bib}

\end{document}