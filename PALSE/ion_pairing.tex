\documentclass[12pt,a4paper,prl,reprint]{revtex4-1}
\usepackage[utf8]{inputenc}
\usepackage{amsmath}
\usepackage{amsfonts}
\usepackage{amssymb}
\usepackage{graphicx}
\usepackage{hyperref}
\usepackage{kpfonts}
\usepackage[version=4]{mhchem}
\usepackage{siunitx}
\usepackage[table]{xcolor}
\usepackage{pgfplots}
\usetikzlibrary{calc}

\begin{document}
\author{Mathieu Leocmach}
\title{Ion pairing rheology}
\maketitle

\section{System description}

We are dealing with low-salt aqueous solutions of polyelectrolytes with cationic moieties along the chain and a single anionic head (Phosphonate, \ce{\bond{-}PO3^2-}). A chain has $N_0 = 70$ units from NMR measurements. The cationic moieties are either all Pyrrolidinium (\ce{Pyr+}) or all Imidazolium (\ce{Im+}) and we note the polymers \ce{PPyr+X-} and \ce{PIm+X-} respectively. The counterion \ce{X-} can be either Fluoride (\ce{F-}), Chloride (\ce{Cl-}), Bromide (\ce{Br-}) or Iodide (\ce{I-}). According to the Hard and Soft Acid and Bases (HSAB) theory, stronger association is expected for harder cation (\ce{Pyr+}) and softer anion (\ce{F-}). Therefore \ce{PPyr+I-} should be the most charged polymer and \ce{PIm+F-} the least charged.

Experimentally, we are able to dissolve the five polymers that are the most charged, but neither the two fluorides nor \ce{PIm+Cl-}. This means that in absence of charge dissociation the chain is in poor solvent. Furthermore, we held the two fluoride polymers in water at boiling temperature overnight without any dissolution. The $\Theta$ temperature is thus higher than \SI{100}{\celsius}. Nonetheless, due to charge dissociation, most of our polymers are soluble at \SI{25}{\celsius}. The reduced temperature $\tau = 1 - T/\Theta$ is thus at least 0.2.

We have previously demonstrated that, due to the opposite charge of the phosphonate head, head-to-body ionic bonds are possible~\cite{Srour2014}. If every chain has a single foreign head attached to its body, every polymer is linked to two others and we obtain a single effective chain of effective polymerisation index $n N_0$ where $n$ is the number of polymers in the effective chain. Here we have supposed that in order to minimize the inter-chain repulsion between charged cationic groups, heads are preferentially attached to the tail of their neighbour. We also neglect the possibility of the two charges of the phosphonate head to attach to two separate chains.

If as least two foreign heads attach to the same body, we obtain an effective cross-linking point. When the probability of such a configuration is non-zero, we obtain a physically cross-linked gel. Indeed, we observe that our 5 soluble polymers form soft solid when mixed with water. Submitted to sufficient shear stress these gels flow but conserve their rheological properties before and after the flow, confirming the physical nature of the gel.

\section{Reological measurements}

Let us now discuss the rheological measurements and link them to microstructure. Amplitude sweep measurements at \SI{1}{\hertz} in a cone-plate geometry give access to two main observables, see Table~\ref{tab:results}: (i) the value of the shear modulus at small amplitude $G^\prime(\gamma\rightarrow 0)$, i.e. the elasticity of the undamaged gel network, and (ii) the critical strain $\gamma_c$ corresponding to the maximum of $G"$, the end of the linear regime and the first broken cross-links.

\begin{table}
\begin{tabular}{l|SS|SSS}
& {$G^\prime(\gamma\rightarrow 0)$} & {$\gamma_c$} & {$n$} & {$E_c$} & {$A$}\\
&	{\si{\pascal}} &  & & {\si{\kilo\joule/\mol}} & \\\hline&&&&\\[-10pt]
\ce{PPyr+I-}	& 8800	&	0.04	&	1.	&	0.004\cellcolor{gray!25}	&	1.\\
\ce{PPyr+Br-}	& 9800 	& 	0.016	&	1.05&	0.001\cellcolor{gray!25}	&	1.1\\
\ce{PPyr+Cl-}	& 270 	&	0.09	&	45	&	0.02\cellcolor{gray!25}	&	7.\\
\ce{PIm+I-}	& 78	&	0.16	&	113	&	0.063\cellcolor{gray!25}	&	9.6\\
\ce{PIm+Br-}	& 13	&	8.04	&	790	&	160.	&	3.4\cellcolor{gray!25}\\
\end{tabular}
\caption{Results. Incoherent values are on gray background.}
\label{tab:results}
\end{table}

\subsection{From modulus to effective chain length}

The spatial conformation of a polyelectrolyte (super)chain in semi-dilute, low salt solution is a persistent random walk where the persistence length is the screening length $r_\mathrm{src}$ due to the dissociated counter ions~\cite{Rubinstein1996}, see Figure~\ref{fig:persistent}. Between each cross-link point the (super)chain can be considered as an entropic spring made of $N=n N_0$ monomers. The shear modulus of the gel is then given by:
\begin{equation}
G = \frac{c}{N}k_\mathrm{B}T,
\label{eq:G}
\end{equation}
where $c$ is the monomer number density, $k_\mathrm{B}$ is the Boltzmann constant and $T$ the temperature. Knowing the molecular mass $M$ of the chains, density $d$ of the solvent and the weight fraction $w$ of polymer, we obtain
\begin{equation}
n = \frac{\mathcal{N_A}}{M} w d \frac{k_\mathrm{B}T}{G}.
\end{equation}

We find that the number $n$ of chains between cross-link point goes from 1 in \ce{PPyr+I-} and \ce{PPy+Br-} to 790 in \ce{PIm+Br-}, following the \textit{a priori} ranking of charge dissociation, see Table~\ref{tab:results}. This is coherent with a higher probability of attaching two or more heads on a highly dissociated body, i.e. a large cross-linking ratio.

\subsection{Bounding energy}

Considering that each super chain is an entropic spring in its linear domain, the mechanical energy stored by a single super chain at a strain $\gamma$ is $E = \gamma^2 k_\mathrm{B}T$. If we suppose that the ionic bonds break before the complete extension of the (super)chains, we evaluate the head-to-body bonding energy to
\begin{equation}
E_c = \gamma_c^2 k_\mathrm{B}T.
\end{equation}

For \ce{PIm+Br-} we measure $\gamma_c = 800\%$ and thus $E_c \approx 65 k_\mathrm{B}T$ per bond or \SI{160}{\kilo\joule\per\mol}, coherent with typical energies of ionic bonds. However for all other samples the critical strain is between 2\% and 20\%, leading to incoherent values for the bonding energies between $10^{-3}k_\mathrm{B}T$ and $10^{-1}k_\mathrm{B}T$ per bond, see Table~\ref{tab:results}. Therefore, the hypothesis of a linear spring fails for these four samples, meaning that the super chain is fully extended at $\gamma_c$.

\begin{figure}
\begin{center}
\begin{tikzpicture}
\begin{axis}[
	scale only axis,
	width=12\baselineskip, height=6\baselineskip, 
	axis lines=none,
	xmin=0.5, xmax=4.5,
	ymin=-0.7, ymax=1.3,
	]
\addplot[only marks, mark=o, mark size=2] table {/home/mathieu/tex/PALSE/persistent.xy} coordinate[pos=0.43] (a) coordinate[pos=0.57] (b) coordinate[pos=0.81](c);
\draw[<->] ($(a)+(-0.5em,0)$) -- ($(b)+(-0.5em,0)$) node[midway, left]{$r_\mathrm{scr}$};
\draw[<->] ($(c)+(-0.25em,-0.5em)$) -- ($(c)+(0.25em,-0.5em)$) node[midway, below]{$D$};
\end{axis}
\end{tikzpicture}
\end{center}
\caption{Sketch of the persistent random walk.}
\label{fig:persistent}
\end{figure}

\subsection{From critical strain to charge dissociation}

For polyelectrolytes in bad solvent the persistent random walk in made of electrostatic blobs of diameter $D$ containing $g_e$ monomers~\cite{Rubinstein1996}, see Figure~\ref{fig:persistent}:
\begin{align}
g_e &= \frac{A^2}{u}\tau\label{eq:ge},\\
D &= b \left(\frac{A^2}{u}\right)^{1/3}\label{eq:D}.
\end{align}
Here $A$ is the number of monomers between uncondensed charges, $\tau$ the reduced temperature and $u = \ell_\mathrm{B}/b \approx 2.7$ is the ratio of Bjerrum length $\ell_\mathrm{B}$ and monomer size $b\approx \SI{0.26}{\nano\metre}$. 

Extending these electrostatic blobs means exposing more monomers to the bad solvent and typically requires energies at least an order of magnitude higher than for the extension of the persistent random walk. We will thus consider that the exit from the linear domain of the material corresponds to the full extension of the persistent random walk:
\begin{equation}
\gamma_c = \frac{R_0}{R} - 1
\end{equation}
where $R$ and $R_0$ are respectively the end-to-end distance and the contour length of the persistent random walk.

A segment of the persistent random walk is $r_\mathrm{scr}$ long and thus contains $g_\mathrm{scr} = g_e r_\mathrm{scr}/D$ monomers. We thus have
\begin{align}
R &= r_\mathrm{scr} \left(\frac{N}{g_\mathrm{scr}}\right)^{1/2},\\
R_0 &= r_\mathrm{scr} \frac{N}{g_\mathrm{scr}},\\
\gamma_c &= \left(\frac{N}{g_\mathrm{scr}}\right)^{1/2} -1\label{eq:gamma0}.
\end{align}

Following~\cite{Rubinstein1996}, we can express the screening length function of $A$ and $\tau$ as
\begin{equation}
r_\mathrm{scr} = \left(\frac{A^2}{u}\right)^{1/3} \left(\frac{\tau}{cb}\right)^{1/2}
\label{eq:rsrc}
\end{equation}

From equations (\ref{eq:ge}), (\ref{eq:D}) and (\ref{eq:rsrc}) we deduce
\begin{equation}
g_\mathrm{scr} = \frac{A^2}{u} \left(\frac{\tau^3}{c b^3}\right)^{1/2}\label{eq:gscr}.
\end{equation}

From equations (\ref{eq:gamma0}) and (\ref{eq:gscr}) we can finally relate the measured critical strain to the microscopic parameter $A$:
\begin{equation}
A = \frac{\left(N u\right)^{1/2}}{\gamma_c+1}\left(\frac{c b^3}{\tau^3}\right)^{1/4}.
\label{eq:A}
\end{equation}

The only unknown here is the reduced temperature $0.2<\tau<1$, possibly different between Poly(pyrrolidinium) based and Poly(immidazolium) based polymers. Setting $\tau\approx 0.235$, corresponding to $\Theta\approx\SI{116}{\celsius}$, yields $A=1$ for both \ce{PPyr+I-} and \ce{PPyr+Br-}, indicating full dissociation of the charges. By contrast \ce{PPyr+Cl-} has a dissociated charge every 7 monomers. Supposing the same $\tau$, \ce{PIm+I-} has a dissociated charge every 10 monomers, see Table~\ref{tab:results}.

\subsection{Charge dissociation and cross-linking ratio}

There must be a link between the charge dissociation and and the amount of physical cross-link. If we assume that the probability for a charge to dissociate is independent of the number of charges already present on the polymer, we can describe the probability of having $k$ charge dissociated on a polymer by a Poisson distribution of mean $\lambda = N_0/A$. Having 2 charges dissociated on a chain implies a cross-link. Therefore, we the cross-link probability $1/n$ is proportional to $\lambda^2$, yielding
\begin{equation}
n \sim A^2.
\end{equation}

This is indeed what we observe, as shown in Figure~\ref{fig:nA}. We can extrapolate this law for the case of \ce{PIm+Br-} and obtain $A = 28$. Inverting equation~(\ref{eq:A}) yields $\gamma_c = 860\%$, indeed higher than the one observed (804\%). This confirms that in the \ce{PIm+Br-} strain sweep we were actually probing the strength of the ionic bonds.

\begin{figure}
\begin{center}
\begin{tikzpicture}
\begin{loglogaxis}[
	xlabel = $A$,
	ylabel = $n$,
	]1.04, 44.88, 0.99, 112.52
	\addplot[only marks] coordinates {(1.090233298, 1.04) (6.98656403, 44.88) (1.000001279, 0.99) (9.589224893, 112.52)};
	\addplot[no marks, domain=1:10] {x^2};
\end{loglogaxis}
\end{tikzpicture}
\end{center}
\caption{Relation between the number of monomers between uncondensed charges and the number of chains between cross-link points. The line corresponds to $A^2$.}
\label{fig:nA}
\end{figure}


\bibliography{PALSE}

\end{document}